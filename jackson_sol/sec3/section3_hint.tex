章末問題の手の付け方,略解とかを書きたい.別の解答もあると思う.

\begin{enumerate}[%
  label=%
  \fbox{%
   {\thesection.\arabic*}%
    },%
    ]
  \item 方位角対称なポテンシャルをLegendre多項式で展開したときの一般形
    \begin{gather}
      \label{eq:potential_in_legendre}
      \Phi(r, \theta) = \sum_{l=0}^\infty \ab(A_l r^l + B_l r^{-(l+1)})P_l(\cos\theta)
    \end{gather}
    に代入して境界条件から各係数を決定する.
    Legendre多項式について成り立つ関係:
    \begin{gather}
      P_n(0) = (-1)^{n/2} 2^{-n} \comb{n}{n/2} \qq (n : \text{even},\, %
      \comb{\,\cdot\,}{\,\cdot\,}\text{は二項係数})
    \end{gather}
    を使えば良い.
  \item 
    \begin{enumerate}[(a)]  
      \item \eqref{eq:potential_in_legendre}
        のように展開して,表面電荷密度による境界条件を考えることで係数を決定する.
        Legendre多項式の直交関係式:
        \begin{gather}
          \int_0^1 \dl{(\cos\theta)} P_{l'}(\cos\theta) P_l(\cos\theta) %
          = \frac{2}{2l+1} \delta_{l'l}
        \end{gather}
        および,Legendre多項式の基本的な式:
        \begin{gather}
          P_l(-x) = (-1)^l P_l(x)
        \end{gather}
        を用いれば良い.
      \item $\bs{E} = - \gr \Phi$を用いて原点での電場を計算する.
        $\hat{\bs{z}} = \cos\theta\hat{\bs{r}} - \sin\theta \hat{\bs{\theta}}$を用いると
        結果を$\hat{\bs{z}}$のみで表すことができる.
      \item (2)では$\beta = \pi - \alpha$として$\beta \to 0$の極限を考えるとよい.
    \end{enumerate}
  \item 
    ちょっと後回しにする.
  \item
    \begin{enumerate}[(a)]  
      \item  
        ポテンシャルを球面調和関数を用いた展開で表す:
        \begin{gather}
          \Phi(r, \theta, \phi) = \sum_{l=0}^\infty \sum_{m=l}^{l} \ab(A_{lm}r^l + B_{lm}r^{-(l+1)}) Y_{lm}(\theta,\phi)
        \end{gather}
        $r=a$の球面上で境界条件から各係数を定める.
        計算の途中で,
        \begin{gather}
          V(\phi) = \left\{
            \begin{aligned}
              +V &\qq \text{if}\qq \frac{\pi}{n}\cdot 2j \leq \phi \leq \frac{\pi}{n}\cdot(2j+1)\\
              -V &\qq \text{if}\qq \frac{\pi}{n}\cdot (2j+1) \leq \phi \leq \frac{\pi}{n}\cdot(2j+2)\\
            \end{aligned}
          \right. \qq (\text{for}\ j = 0, 1, \cdots, n-1)
        \end{gather}
        を用いる.また,積分
        \begin{gather}
          \int_0^{2\pi} \dl{\phi} V(\phi) \e^{-\i m\phi}
        \end{gather}
        の計算があらわれるが,場合わけをして,$m \neq 0$のときを考えると,
        \begin{gather}
          \int_0^{2\pi} \dl{\phi} V(\phi) \e^{-\i m\phi} =%
          -\frac{\i V}{m}\ab[1 - \exp\ab(-\i \frac{m}{n}\pi)]^2 %
          \sum_{j=0}^{n-1} \exp\ab(-\i\frac{m}{n}2j\pi)
        \end{gather}
        となる.

        まず,$m / n$が整数となるときを考える.
        $m = 0, \pm 2n, \pm 4n, \ldots$のときは
        $\exp\ab(-\i(m/n)\pi) = 1$となるので,積分はゼロ.
        残りとして考えるのは$m = \pm n, \pm 3n, \ldots$である.
        このとき,
        \begin{gather}
          \exp\ab(-\i \frac{m}{n}2j \pi) = 1 \qq \text{for}\qq j = 0, \ldots, n-1
        \end{gather}
        である.

        次に,$m / n$が整数とならない時を考える.
        \begin{gather}
          \sum_{j=0}^{n-1} \exp\ab(-\i \frac{m}{n}2j\pi) %
          = \frac{1 - \exp(-\i 2m\pi)}{1 - \exp\ab(-\i \frac{2m}{n}\pi)} = 0
        \end{gather}
        である($m$は整数であるので,分子はゼロ)から,この場合は積分がゼロになる.

        これらをまとめて
        \begin{gather}
          \int_0^{2\pi} \dl{\phi} V(\phi) e^{-\i m \phi} = 
          \begin{dcases}
            - \i\frac{4Vn}{m} & \text{if}\qq m = \pm n, \pm 3n, \ldots\\
            \qq 0 &\text{otherwise}
          \end{dcases}
        \end{gather}
        と書ける.
      \item (a)の結果を用いて具体的に計算を進めていくだけであるが,
        Jackson §3.3 eq. (3.36)との比較の時には座標軸の取り方に気をつける必要がある.
        具体的には,$\cos\theta' = \sin\theta \sin \phi$とすれば良い.
    \end{enumerate}
  \item Jackson eq. (3.70)の式を$r, a$でそれぞれ微分して
    ,差を考えて$\dl{\Omega'}V(\theta' , \phi')$で積分をすれば良い.
  \item ポテンシャルが具体的に計算できるので,
    Jackson eq. (3.70)を用いて,球面調和関数で展開して,和を考えれば良い.
    この問題は易しい.
  \item 前問と同じように考えれば良い.
  \item $\log(\csc\theta) = \log(1/\sin\theta)$をLegendre多項式で展開する.このときに積分
    \begin{gather}
      \int \dl{x} \log(1 - x^2) = (1+x)\log(1+x) - (1-x)\log(1-x) - 2x
    \end{gather}
    を用いる.普通に$\log$積分をしてしまうと発散してしまうことに注意.
    この積分は知らないと難しいかも.
  \item 円筒座標系でのLaplace方程式
    \begin{gather}
      \frac{1}{\rho}\diffp**{\rho}{\ab(\rho \diffp{\Phi}{\rho})} +%
      \frac{1}{\rho^2}\diffp[2]{\Phi}{\phi} + \diffp[2]{\Phi}{z} = 0
    \end{gather}
    を$z = 0, L$での境界条件に注意して解けば良い.
  \item 前問の結果を用いればよい.(b)での極限を考える時は,
    \begin{gather}
      I_\nu(z) \sim \frac{1}{\nu !} \ab(\frac{z}{2})^\nu %
      + \order{z^{\nu+1}} \qq \text{if} \qq z \ll 1
    \end{gather}
    を用いると良い.また,三角関数の無限和を求める時は,
    $\exp$の形にして無限級数として和を求めてその実部・虚部を求めると見通しが良い.
    また,$\log$の虚部は$\arg$で与えられることに注意.
  \item %3.11
  \item %3.12
    円筒座標系でLaplace方程式を解く.境界条件は
    \begin{gather}
      V(\rho, z) = V \theta(a - \rho)\delta(z)
    \end{gather}
    で与えられる.
    また,公式
    \begin{gather}
      \int_0^\infty \dl{x} \e^{-\alpha x} J_0(bx)  = \frac{1}{\sqrt{\alpha^2 + b^2}}\\
      \int_0^\infty \dl{x} \e^{-\alpha x} \ab[J_0(bx)]^2 %
      = \frac{2}{\pi \sqrt{\alpha^2 + 4b^2}}K\ab(\frac{2b}{\sqrt{\alpha^2 + 4b^2}})
    \end{gather}
    を用いる.
    ここで,$K(k)$は,第一種完全楕円積分
    \begin{gather}
      K(k) = \int_0^{\pi/2} \frac{\dl{\theta}}{\sqrt{1-k^2\sin^2\theta}}
    \end{gather}
    である.
  \item %3.13
    Jackson eq. (3.125)のGreen関数の表式を用いる.
    \begin{gather}
      \int_0^1 \dl{x}  P_l(x) = 
      \begin{dcases}
        1 \qqtext{if} l = 0\\
        \frac{(-1)^k(2k-1)!!}{2^{k+1}(k+1)!} \qqtext{if} l = 2k+1; k = 0, 1, \ldots\\
        0 \qqtext{if} l = 2k; k = 1, 2, \ldots
      \end{dcases}
    \end{gather}
    に注意.丁寧に計算を進めるとproblem 3.1と同じ結果を得る.
  \item %3.14
    線電荷密度を求めて,それを体積電荷密度で書くと
    \begin{gather}
      \rho_{\mr{c}} (\bs{x}) = \frac{3Q}{8\pi d^3} \frac{d^2 - r^2}{r^2} %
      \ab[\delta(\cos\theta - 1) + \delta(\cos\theta + 1)]
    \end{gather}
    と表すことができる.Jackson eq. (3.125)のGreen関数を用いて
    球内部のポテンシャルを求めればよい.$r \gtrless d$で積分の計算が異なるが,
    丁寧に計算をすれば良い.
  \item %3.15
  \item %3.16
    \begin{enumerate}[(a)]
      \item
        Besselの微分方程式
        \begin{gather}
          \diff[2]{J_\nu(k\rho)}{\rho} + \frac{1}{\rho} \diff{J_\nu(k\rho)}{\rho} %
          + \ab(k^2 - \frac{\nu^2}{\rho^2})J_\nu(k\rho)  = 0
        \end{gather}
        から出発して,部分積分,$k, k'$の入れ替えをして差を考えると,
        \begin{gather}
          (k^2 - (k')^2) \int_0^\infty \dl{\rho} \rho J_\nu(k\rho)J_\nu(k'\rho) 
          = \ab[\rho J_\nu(k\rho)\diff{J_\nu(k'\rho)}{\rho} %
          - \rho J_\nu(k'\rho)\diff{J_\nu(k\rho)}{\rho}]_{\rho=0}^{\rho=\infty}
        \end{gather}
        となる.$\rho = 0$では,$\Re (\nu) > -1$の時にゼロとなる.
        $\rho = \infty$の場合を考える.$\rho = R$として,$R \to \infty$を考えることにする.
        Bessel関数の漸近形を用いて,
        \begin{gather}  
          \int_0^\infty \dl{\rho}\rho J_\nu(k\rho) J_\nu(k' \rho) 
          = \lim_{R\to \infty}\frac{1}{\pi}\frac{1}{\sqrt{kk'}} %
          \ab[-\frac{\cos[(k+k')R - \nu \pi]}{k+k'} + \frac{\sin\ab[(k-k')R]}{k-k'}]
        \end{gather}
        と書き直すことができる.

        さて,デルタ関数は,$\sinc$関数を用いて
        \begin{gather}
          \lim_{\epsilon \to 0} \delta_\epsilon(x) %
          = \lim_{\epsilon \to 0} \frac{\sin(x/\epsilon)}{x / \epsilon} %
          \frac{1}{\pi \epsilon} = \delta(x)\\
          \lim_{\epsilon \to 0} \frac{\cos(x/\epsilon)}{x/\epsilon} %
          \frac{1}{\pi \epsilon} = 0
        \end{gather}
        として表すことができることに注意すると,
        \begin{gather}
          \lim_{R\to \infty} \frac{1}{\pi}%
          \frac{1}{\sqrt{kk'}}\ab[\frac{\sin\ab[(k-k')R]}{k-k'}]  %
          = \frac{1}{k}\delta(k-k')\\
          \begin{split}
            &\lim_{R \to \infty} \frac{1}{\pi} %
            \frac{1}{\sqrt{kk'}} \frac{\cos\ab[(k+k')R - \nu \pi]}{k+k'} \\
            &\qq\qq\qq  = \frac{1}{\sqrt{kk'}} \delta(k+k')\cos(\nu \pi) %
            = 0 \qq \text{if}\qq k, k'>0
        \end{split}
        \end{gather}
        であるから,
        \begin{gather}
          \int_0^\infty \dl{\rho}\rho J_\nu(k\rho) J_\nu(k' \rho) = \frac{1}{k}\delta(k-k')
        \end{gather}
        が従う.
      \item 基本的には§3.11の議論を追えば良い.
      \item (b)の結果を用いるだけ.
      \item Besselの積分表示
        \begin{align}
          J_n(x) &= \frac{1}{\pi} \int_0^\pi \dl{\theta}\cos\ab(n\theta - x \sin\theta)\\
                 &=  \frac{1}{2\pi} %
                 \int_0^{2\pi} \dl{\theta} \cos\ab(n \theta - x \sin\theta)
        \end{align}
        Hansenの積分表示
        \begin{gather}
          J_n(x) = \frac{1}{\i^n \pi} \int_0^\pi \dl{\theta} %
          \e^{\i x \cos\theta} \cos (n\theta)
        \end{gather}
    \end{enumerate}

  \item %3.17
    \begin{enumerate}[(a)]  
      \item Fourier展開(・半区間展開)を用いると,
        \begin{gather}
          \delta(\phi - \phi') = \frac{1}{2\pi} \sum_{m=-\infty}^{\infty} %
          \e^{\i m (\phi - \phi')}\\
          \delta(z- z') = \frac{2}{L} %
          \sum_{n=1}^\infty \sin\ab(\frac{n\pi z}{L}) \sin\ab(\frac{n \pi z'}{L})
        \end{gather}
        と展開できる.
      \item problem 3.16 (a)の結果を用いて,$\rho$方向と$\phi$方向について展開をした後,
        $z$方向についてGreen関数の満たすべき条件を考える.
    \end{enumerate}
  \item %3.18
    \begin{enumerate}[(a)]  
      \item  円筒座標系でのLaplace方程式を境界条件
        \begin{gather}
          \begin{dcases}
            \Phi = 0 \qq \text{on}\, z = 0\\
            \Phi = V\theta(a - \rho) \qq \text{on} \, z = L
          \end{dcases}
        \end{gather}
        で解く.
      \item Bessel関数,$\sinh$関数を級数展開して,最低次の寄与を計算する.
      \item (b)と同様.
    \end{enumerate}
  \item %3.19
    Gradshteyn \& Ryzhik, p.722 eq. (6.666)より
    \begin{gather}
      \int_0^\infty \dl{x} x^{\nu+1} %
      \frac{\sinh(\alpha x)}{\sinh(\pi x)} J_\nu(\beta x) %
      = \frac{2}{\pi}\sum_{n=1}^\infty (-1)^{n-1} n^{\nu+1} %
      \sin(n \alpha) K_\nu(n \beta)
      \qq\ab(\text{for}\,\ab|\Re(\alpha)| < 1, \Re(\nu) > -1)
    \end{gather}
    である.
  \item %3.20
    problem 3.17で考えたGreen関数を用いてポテンシャルを計算することができる.
  \item %3.21
    problem 1.18(b)での結果を用いる.
    また,Gradshteyn \& Ryzhik eq. (6.554)より
    \begin{gather}  
      \int_0^1 \dl{x} \frac{x}{\sqrt{1-x^2}}J_0(xy) = \frac{\sin y}{y} \qq\text{for}\qq y > 0
    \end{gather}
    である.

    (c)で$d \gg R$のとき,変分法を使って求める方法がわからない...
  \item %3-22
    二次元のGreen関数を求める問題.$\phi = 0, \beta$で
    ゼロとなる境界条件に注意して考えれば良い.
  \item %3-23

\end{enumerate} 



