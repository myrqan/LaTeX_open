\begin{enumerate}[label={\large 3.\arabic*}]
  \item 方位角対称なポテンシャルをLegendre多項式で展開したときの一般形
    \begin{gather}%
      \label{eq:legendre_expand}
      \Phi(r,\theta) = \sum_{l=0}^\infty \ab(A_l r^l + B_l r^{-(l+1)}) P_l\ab(\cos\theta)
    \end{gather}%
    に代入して,境界条件から各係数$A_l ,B_l$を決定する.
    また,偶数次のLegendre多項式について,
    \begin{gather}%
      P_n(0) = (-1)^{n/2} 2^{-n} \comb{n}{n/2}
    \end{gather}
    が成り立つ.
  \item 
    \begin{enumerate}[(a)]%  
      \item  
        \eqref{eq:legendre_expand}のように展開して,表面電荷密度による
        境界条件を考えれば良い.
        また,Legendre多項式の直交関係式
        \begin{gather}%
          \int_0^1 \dl{x} P_{l'}\ab(x) P_l\ab(x) = \frac{2}{2l+1} \delta_{l'l}
        \end{gather}%
        および負の引数をとる場合の式
        \begin{gather}%
          P_l(-x) =  (-1)^l P_l(x)
        \end{gather}%
        を用いれば良い.
      \item $\bs{E} = -\gr \Phi$を用いて原点での電場を計算できる.
        また,$\hat{\bs{z}} = \cos\theta \hat{\bs{r}} - \sin\theta \hat{\bs{\theta}}$を
        用いると,結果を$\hat{\bs{z}}$のみで表すことができる.
      \item (2)では$\beta = \pi - \al$として$\beta \to 0$の極限を考えるほうがやりやすい.
    \end{enumerate}
  \item 後で
  \item 
    \begin{enumerate}[(a)]%  
      \item  ポテンシャルを球面調和関数で展開する.
        \begin{gather}%
          \Phi(r,\theta,\phi) = \sum_{l=0}^\infty \sum_{m=-l}^{l} \ab(A_{lm} r^l + B_{lm}r^{-(l+1)}) Y_{lm}(\theta,\phi)
        \end{gather}%
        である.境界条件は$r=a$の球面上で考えれば良い.$\phi$の値でポテンシャルを場合分
        けして考えると見通しが良い.
      \item (a)の結果を用いて具体的な計算をするだけ.本文\S3.3中式(3.36)との比較のときには,
        座標軸の取り方に注意をする必要がある.
        具体的には$\cos\theta' = \sin\theta \sin\phi$である.
    \end{enumerate}%
  \item 本文中の式(3.70)を$r,a$でそれぞれ微分,差を考えて$\dl{\Omega'} V(\theta',\phi')$
    で積分をすればよい.
  \item ポテンシャルが具体的に計算できるので,本文中の式(3.70)を使って球面調和関数で展開し
    た後に,和を考えれば良い.
  \item 前問と同じように考えれば良い.
  \item $\log(\csc\theta) = \log(1/\sin\theta)$をLegendre多項式で展開する.
    積分
    \begin{gather}%
      \int \dl{x} \log(1-x^2) = (1+x)\log(1+x) - (1-x)\log(1-x) - 2x
    \end{gather}%
    を使うと見通しが良い.
  \item 円筒座標系でのLaplacianは
    \begin{gather}%
      \gr^2 = \frac{1}{\rho}\diffp**{\rho}{\ab(\rho\diffp{}{\rho})} + \frac{1}{\rho^2} \diffp[2]{}{\phi} + \diffp[2]{}{z}
    \end{gather}%
    である.これを$z=0,L$での境界条件に注意して解けばよい.
  \item 前問の結果を用いる.(b)での極限を考えるときは,漸近形
    \begin{gather}%
      I_\nu(z) \sim \frac{1}{\nu !}\ab(\frac{z}{2})^\nu + \order{z^{\nu+1}} \qqtext{for} z \ll 1
    \end{gather}%
     を用いるとよい.三角関数の無限和は$\exp$での無限級数として考えると見通しが良い.
   \item あとで
   \item Bessel関数に関する公式
     \begin{gather}%
       \int_0^\infty \dl{x} \e^{-\alpha x} J_0(bx) = \frac{1}{\sqrt{\alpha^2 + b^2}}\\
       \int_0^\infty \dl{x} \e^{-\alpha x} \ab[J_0(bx)]^2 = \frac{2}{\pi \sqrt{\alpha^2 + 4b^2}} 
       K\ab(\frac{2b}{\sqrt{\alpha^2 + 4b^2}})
     \end{gather}%
     が知られている.ここで,$K(k)$は第一種完全楕円積分
     \begin{gather}%  
       K(k) = \int_0^{\pi/2} \frac{\dl{\theta}}{\sqrt{1-k^2\sin^2\theta}}
     \end{gather}%
     である.
\end{enumerate}
