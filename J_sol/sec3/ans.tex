\toi{3.1}
\begin{wrapfigure}[9]{r}{0.4\linewidth}%
  \centering
  \includegraphics[width=\linewidth]{fig/Jackson3-1.pdf}%  
\end{wrapfigure}
半径$a, b$の球の間の領域のポテンシャルを考える.Legendre多項式による展開
\begin{gather}%
  \Phi(r, \theta) = \sum_{l=0}^\infty \ab(A_l r^l + B_l r^{-(l+1)}) P_l (\cos\theta)
\end{gather}
を以下の境界条件で考える:
\begin{gather}%  
  \label{eq:prob3-1_bc1}
  \Phi(a,\theta) =
  \begin{dcases}%
    V \qqtext{if} 0 \leq \theta \leq \frac{\pi}{2}\\
    0 \qqtext{if} \frac{\pi}{2} \leq \theta \leq \pi
  \end{dcases},\\
  \label{eq:prob3-1_bc2}
  \Phi(b,\theta) =
  \begin{dcases}%
    0 \qqtext{if} 0 \leq \theta \leq \frac{\pi}{2}\\
    V \qqtext{if} \frac{\pi}{2} \leq \theta \leq \pi
  \end{dcases}% 
\end{gather}%
\eqref{eq:prob3-1_bc1}について,$P_{l'}(\cos\theta)\dl{(\cos\theta)}$をかけて$\theta\in[0,\pi]$
で積分をすると,
\begin{multline}%
  \int_{\cos\theta=1}^{\cos\theta=-1} \sum_{l=0}^\infty\ab(A_la^l + B_la^{-(l+1)})
  P_l(\cos\theta) P_{l'}(\cos\theta) \dl{(\cos\theta)} \\
  =  V \int_{\cos\theta=1}^{\cos\theta=0} P_{l'}(\cos\theta) \dl{(\cos\theta)}
\end{multline}%
である.左辺の積分変数を$x$で書くと,
\begin{gather}%
  \text{(L.H.S.)} = \int_{1}^{-1} \sum_{l=0}^\infty\ab(A_l a^l + B_l a^{-(l+1)})
  P_l(x) P_{l'}(x) \dl{x}
\end{gather}
であり,Legendre多項式に関する直交関係\eqref{eq:legendre_orthogonal}より,($l,l'$を入れ替えて)
\begin{gather}%
  \label{eq:prob3-1_a}
  A_l a^l + B_l a^{-(l+1)} = \frac{V(2l+1)}{2} \int_0^1 \dl{x} P_l(x)
\end{gather}%
を得る.
同様に考えて,\eqref{eq:prob3-1_bc2}について,
\begin{gather}%
  \label{eq:prob3-1_b}
  A_l b^l + B_l b^{-(l+1)} =(-1)^l \frac{V(2l+1)}{2} \int_0^{1} \dl{x} P_l(x)
\end{gather}%
が成り立つ.
ただし,$P_l(-x) = (-1)^l P_l(x)$に注意せよ.

式\eqref{eq:prob3-1_a}と\eqref{eq:prob3-1_b}より,係数$A_l, B_l$を計算することができて,
\begin{gather}%
  A_l = \frac{(-1)^lb^{l+1}-a^{l+1}}{b^{2l+1}-a^{2l+1}}
  \frac{V(2l+1)}{2} \int_0^1 \dl{x} P_l(x),\\
  B_l = \frac{b^l-(-1)^la^l}{b^{2l+1}-a^{2l+1}}a^{l+1}b^{l+1}
  \frac{V(2l+1)}{2}\int_0^1 \dl{x} P_l(x)
\end{gather}%
がわかる.
したがって,ポテンシャルの正確な形は
\begin{multline}
  \Phi(r,\theta) =
  V \sum_{l=0}^\infty \ab(
  \frac{(-1)^lb^{l+1}-a^{l+1}}{b^{2l+1}-a^{2l+1}} r^l +
  \frac{b^l-(-1)^la^l}{b^{2l+1}-a^{2l+1}}a^{l+1}b^{l+1} r^{-(l+1)}
  )\\
  \times\frac{2l+1}{2}\ab(\int_0^1 \dl{x} P_l(x))
  P_l(\cos\theta)
\end{multline}

である.
具体的に$l = 4$の項までを書き下せば,
\begin{multline}
  \Phi(r,\theta) = V \left[
    \frac{1}{2} + \frac{3}{4}\frac{-(a^2+b^2)r+a^2b^2(a+b)r^{-2}}{b^3-a^3} P_1(\cos\theta)\right.\\
    \left.
    -\frac{7}{16}\frac{-(a^4+b^4)r^3 + a^4b^4(a^3+b^3)r^{-4}}{b^7-a^7}P_3(\cos\theta) +\cdots
    \right]
\end{multline}
となる.

$b \to \infty$の極限では$a/b \to 0, r/b\to 0$として,
\begin{align}%
  \Phi(r,\theta) 
  &= V
  \sum_{l=0}^\infty \frac{2l+1}{2}\ab(\frac{a}{r})^{l+1}
  \ab(\int_0^1\dl{x} P_l(x)) P_l(\cos\theta)\notag\\
  &= V\ab[
    \frac{1}{2} + \sum_{k=0}^\infty \frac{(-1)^k(4k+3)(2k-1)!!}{2^{k+2}(k+1)!} 
    \ab(\frac ar)^{2k+2} P_{2k+1}(\cos\theta)
  ]\\
  &= V\ab[
    \frac{1}{2} + \frac{3}{4}\ab(\frac{a}{r})^2 P_1(\cos\theta) -
    \frac{7}{16}\ab(\frac{a}{r})^4 P_3(\cos\theta) + \cdots
  ]
\end{align}%
である.
ただし,ここでは$(-1)!! = 1$と約束する.
これは半径$a$の導体球について,
北半球のポテンシャルが$V$,南半球が接地されており,さらに
無限遠でのポテンシャルの境界条件が$V/2$となる系において,導体球の外部領域に作る
ポテンシャルに等しいことがわかる.

$a \to 0$の極限では
\begin{align}%
  \Phi(r,\theta) &= V \sum_{l=0}^\infty (-1)^l \frac{2l+1}{2} \ab(\frac rb)^{l} 
  \ab(\int_0^1\dl{x} P_l(x)) P_l(\cos\theta)\notag\\
  &= V\ab[\frac 12 + \sum_{k=0}^\infty \frac{(-1)^{k+1}(4k+3)(2k-1)!!}{2^{k+2}(k+1)!}
  \ab(\frac rb)^{2k+1} P_{2k+1} (\cos\theta)]\\
  &= V\ab[\frac 12 - \frac 34 \ab(\frac rb) P_1(\cos\theta) +
  \frac{7}{16}\ab(\frac rb)^3P_3(\cos\theta)]
\end{align}%
である.
これは,半径$b$の導体球について,
北半球が接地されていて,南半球のポテンシャルが$V$となるような系における,
球内部のポテンシャルに等しいことがわかる.

それぞれの場合について,ポテンシャルを描く
と図\ref{fig:3-1_lima},\ref{fig:3-1_limb},\ref{fig:3-1_all}のようになる\footnote{
  描画のときに使用したコードは\texttt{.ipynb}形式で保存してある.
  }.
  \begin{figure*}[htbp]%  
    \centering%  
    \begin{minipage}{0.30\linewidth}
      \centering
      \includegraphics[width=\linewidth]{py/3-1_lima.png}%  
      \caption{$a \to 0$としたとき}%  
      \label{fig:3-1_lima}%  
    \end{minipage}
    \begin{minipage}{0.30\linewidth}
      \centering
      \includegraphics[width=\linewidth]{py/3-1_limb.png}%  
      \caption{$b\to\infty$としたとき}%  
      \label{fig:3-1_limb}%  
    \end{minipage}
    \begin{minipage}{0.30\linewidth}
      \centering
      \includegraphics[width=\linewidth]{py/3-1_all.png}
      \caption{一般の場合についてのポテンシャル}
      \label{fig:3-1_all}
    \end{minipage}
  \end{figure*}%


  \hrulefill\\
  %--- 問3.2 ---%
  \toi{3.2}
  \begin{wrapfigure}[10]{r}{0.35\linewidth}
    \centering%  
    \includegraphics[width=\linewidth]{fig/Jackson3-2.pdf}%  
  \end{wrapfigure}%
  極角$\theta$が$\theta < \alpha$を満たす部分には電荷が存在せず,その他の部分には
  一様な電荷密度$\sigma_{\mr{c}} = Q /(4\pi R^2)$が存在するような球を考える.

\itemlabel{(a)}
  ポテンシャルをLegendre多項式で展開して
  \begin{gather}%
    \Phi(r,\theta) = \sum_{l=0}^\infty \ab(A_l r^l + B_l r^{-(l+1)}) P_l(\cos\theta)
  \end{gather}%
  とする.
  球の内部で$r\to 0$としたときに発散しない条件より$l = 0, 1, \ldots$について$B_l = 0$
  また,球の外部については$r \to \infty$としたときにポテンシャルがゼロに漸近するとして,
  $l = 0, 1, \ldots$について$A_l = 0$である.
  したがって,球の内外でのポテンシャル$\Phi_{\text{in}}$および$\Phi_{\text{out}}$は
  \begin{gather}%
    \Phi_{\text{in}}(r,\theta) = \sum_{l} A_l r^l P_l(\cos\theta),\\
    \Phi_{\text{out}}(r,\theta) = \sum_{l} B_l r^{-(l+1)} P_l(\cos\theta)
  \end{gather}%
  とかける.
  ポテンシャルは$r= R$の球面上で連続であるから,
  \begin{gather}%
    \sum_l A_l R^l P_l(\cos\theta) = \sum_l B_l R^{-(l+1)}P_l(\cos\theta)
  \end{gather}%
  であり,$\ab\{P_l\}_{l=0,1,2,\ldots}$が直交関数系をなすことより
  $B_l = A_l R^{2l+1}$である.

  また,電場の接続条件は
  \begin{gather}%
    \sigma = \eps_0 \ab[\eval{\diffp{\Phi_{\text{in}}}{r}}_{r=R} -
    \eval{\diffp{\Phi_{\text{out}}}{r}}_{r=R}] 
    =
    \begin{dcases}%
      \frac{Q}{4\pi R^2} \qqtext{if} \theta \geq \alpha\\
      0 \qqtext{if} 0 \leq \theta < \alpha
    \end{dcases}%
  \end{gather}%
  である.
  \begin{gather}%
    \eval{\diffp{\Phi_{\text{in}}}{r}}_{r=R}
    - \eval{\diffp{\Phi_{\text{out}}}{r}}_{r=R} 
    = \sum_{l=0}^\infty A_l \ab(2l+1) R^{l-1} P_l(\cos\theta)
  \end{gather}%
  に注意すると,$\dl{(\cos\theta)}P_{l'}(\cos\theta)$をかけて$\cos\theta =1$から
  $\cos\theta = -1$で積分をすると,
  直交関係より以下が得られる($l,l'$の入れ替えを行った);
  \begin{gather}%
    A_l = \frac{Q}{8\pi \eps_0}\frac{1}{R^{l+1}} \int_{-1}^{\cos\alpha} \dl{x} P_l(x).
  \end{gather}%
  Legendre多項式に関する漸化式\eqref{eq:legedre_zenka1}
  より,
  \begin{gather}%
    \int_{-1}^{\cos\alpha} \dl{x} P_l(x) 
    = \frac{1}{2l+1}\ab[P_{l+1}(\cos\alpha) - P_{l-1}(\cos\alpha)]
  \end{gather}%
  である($P_{l+1}(-1) - P_{l-1}(-1) = 0$であることに注意せよ).
  したがって,
  \begin{gather}%
    A_l = \frac{Q}{8\pi \eps_0} \frac{1}{2l+1} \ab[P_{l+1}(\cos\alpha) - P_{l-1}(\cos\alpha)]\frac{1}{R^{l+1}}
  \end{gather}%
  であり,球内部のポテンシャルは
  \begin{gather}%
    \Phi_{\text{in}}(r,\theta) = \frac{Q}{8\pi \eps_0} \sum_{l=0}^\infty \frac{1}{2l+1}
    \ab[P_{l+1}(\cos\alpha) - P_{l-1}(\cos\alpha)] \frac{r^l}{R^{l+1}} P_l(\cos\theta)
  \end{gather}%
  と表される.
  また,球外部のポテンシャルについては,
  \begin{gather}%  
    \Phi_{\text{out}}(r,\theta) = \frac{Q}{8\pi \eps_0} \sum_{l=0}^\infty \frac{1}{2l+1}
    \ab[P_{l+1}(\cos\alpha) - P_{l-1}(\cos\alpha)] \frac{R^l}{r^{l+1}} P_l(\cos\theta)
  \end{gather}%
  と表される.

\itemlabel{(b)}
  電場は$\bs{E} = -\gr \Phi$で与えられるので,
  \begin{align}%  
    E_r &= -\diffp{\Phi}{r} \notag\\
    &= -\frac{Q}{8\pi \eps_0}\sum_{l=1}^\infty
    \frac{l}{2l+1}\ab[P_{l+1}(\cos\alpha) - P_{l-1}(\cos\alpha)] 
    \frac{r^{l-1}}{R^{l+1}} P_l(\cos\theta)\\
    E_\theta &= -\frac{1}{r}\diffp{\Phi}{\theta} \notag\\
    &= -\frac{1}{r}\frac{Q}{8\pi\eps_0} 
    \sum_{l=0}^\infty \frac{1}{2l+1} \ab[P_{l+1}(\cos\alpha) - P_{l-1}(\cos\alpha)]\frac{r^l}{R^{l+1}} \diff{P_l(\cos\theta)}{\theta}
  \end{align}%
  である.
  原点では$r = 0$として考えると,$E_r, E_\theta$ともに$l=1$の項だけが残るので,
  $P_2(\cos\alpha) - P_0(\cos\alpha) = -(3/2)\sin^2\al$であることなどに注意すると,
  \begin{align}%
    \bs{E}(r=0) &= \frac{Q}{16\pi\eps_0}\frac{1}{R^2}\sin^2\alpha \ab[\cos\theta \hat{\bs{r}} - \sin\theta \hat{\bs{\theta}}] \notag\\
    &= \frac{Q}{16\pi \eps_0} \frac{1}{R^2}\sin^2\al \hat{\bs{z}}
  \end{align}%
  となる.

\itemlabel{(c-1-a)}
  $\al \to 0$の極限を考える.

  $\cos\al \sim 1 - \al^2 / 2$であるから,$P_l(\cos\al) \sim P_l(1) - (\al^2/2)P_l'(1)$として
  Taylor展開できる.
  $l = 0$のときは
  \begin{gather}%
    P_1(\cos\al) - P_{-1}(\cos\al) = \cos\al + 1 \sim 2 - \frac{\al^2}{2}
  \end{gather}%
  $l \geq 1$のときは
  \begin{align}%
    P_{l+1}(\cos\al) - P_{l-1}(\cos\al) 
    &\sim -\frac{\al^2}{2}\ab(P_{l+1}'(1) - P_{l-1}'(1))\notag\\
    &= -\frac{\al^2}{2}\ab(2l+1) P_l(1) = -\frac{2l+1}{2}\al^2
  \end{align}%
  と近似されるので,
  \begin{align}%
    \Phi_{\text{in}} &\sim 
    \frac{Q}{4\pi \eps_0} \ab[\frac{1}{R} 
    - \frac{\al^2}{4}\sum_{l=0}^\infty \frac{r^l}{R^{l+1}}P_l(\cos\theta)]\notag\\
    &= \frac{Q}{4\pi\eps_0}\ab[
      \frac{1}{R} - \frac{\al^2}{4} \frac{1}{\ab|\bs{x} - R\hat{\bs{z}}|}
    ]
  \end{align}%
  とかける.
  また,球外部のポテンシャルは
  \begin{align}%
    \Phi_{\text{out}} &\sim \frac{Q}{4\pi\eps_0}\ab[
      \frac{1}{r} - \frac{\al^2}{4} \sum_{l=0}^\infty \frac{R^l}{r^{l+1}}P_l(\cos\theta)
    ]\notag\\
    &= \frac{Q}{4\pi\eps_0} \ab[
      \frac{1}{r} - \frac{\al^2}{4} \frac{1}{\ab|\bs{x} - R\hat{\bs{z}}|}
    ]
  \end{align}%
  であるから,球内外のポテンシャルは
  \begin{gather}%
    \Phi \sim \frac{Q}{4\pi\eps_0} \ab[\frac{1}{r_>} - \frac{\al^2}{4}\frac{1}{\ab|\bs{x} - R \hat{\bs{z}}|}]
  \end{gather}%
  と統一的に書くことができる.ただし,$r_>$は$r$と$R$のうち大きい方である.

\itemlabel{(c-1-b)}
  中心における電場の極限は$\sin\al \sim \al$に注意して
  \begin{gather}%
    \bs{E}(r = 0) \sim \frac{Q}{16\pi \eps_0}\frac{\al^2}{R^2} \hat{\bs{z}}
  \end{gather}%
  である.

\itemlabel{(c-2-a)}
  $\al \to \pi$の極限を考えるが,簡単のため$\beta \equiv \pi - \al \to 0$の極限を考える.
  \begin{gather}%
    \cos\al = \cos(\pi -\be) \sim -1 + \frac{\be^2}{2}
  \end{gather}%
  に注意すると,(c-1)と同様にして,
  \begin{gather}%
    P_{l+1}(\cos\al) - P_{l-1}(\cos\al) \sim (-1)^l \frac{2l+1}{2} \be^2
    \qqtext{for} l = 0, 1, \ldots
  \end{gather}%
  と表せる.したがって,
  \begin{align}%
    \Phi_{\text{in}} &\sim \frac{Q}{8\pi\eps_0} \be^2
    \frac{1}{2} \sum_{l=0}^\infty (-1)^l \frac{r^l}{R^{l+1}} P_l(\cos\theta)\notag\\
    &= \frac{Q}{16\pi\eps_0}\beta^2 \sum_{l=0}^\infty \frac{r^l}{R^{l+1}} P_l(-\cos\theta)\notag\\
    &= \frac{Q}{16\pi\eps_0} \beta^2 \frac{1}{\ab|\bs{x} + R \hat{\bs{z}}|}
  \end{align}%
  である.球外のポテンシャルも同様に計算することができて,
  \begin{gather}%
    \Phi_{\text{out}} \sim \frac{Q}{16\pi\eps_0} \beta^2 \frac{1}{\ab|\bs{x} + R \hat{\bs{z}}|}
  \end{gather}%
  となる(同じ結果を与える).

\itemlabel{(c-2-b)}
  原点での電場は$\sin\al =\sin\be \sim \be$に注意して,
  \begin{gather}%  
    \bs{E}(r=0) \sim \frac{Q}{16\pi\eps_0} \frac{\be^2}{R^2} \hat{\bs{z}}
  \end{gather}%
  と計算される.

  \begin{tcolorbox}[dashedbox]%  
    (c-1)での結果について,これは,半径$R$の球表面に総電荷$Q$が分布しており,極角$\al$の部分には
    それに加えて,表面電荷密度$-\sigma = -Q/(4\pi R^2)$が分布している状況と一致している.
    具体的に計算をすると,まず総電荷$Q$が一様に分布しているとき,球内部のポテンシャルは
    簡単な計算により
    \begin{gather}%
      \Phi_{\text{in}}^{(1)} = \frac{Q}{4\pi\eps_0} \frac{1}{R}
    \end{gather}%
    であることがわかる.次に,極角$\al$の範囲については,$\al$が十分に小さいとき,
    この部分に存在する電荷は
    \begin{align}%
      \int \dl{\Omega}R^2(-\sigma) &= -2\pi R^2\sigma \int_0^\al \dl{\theta} \sin\theta\notag\\
      &= -2\pi R^2\sigma \ab(1-\cos\al) \notag\\
      &\sim -2\pi R^2 \sigma \frac{\al^2}{2} = -\frac{\al^2}{4}Q
    \end{align}%
    である.$\al$が十分に小さいとして,この電荷を点電荷とみなすことにすると
    これが作るポテンシャルは
    \begin{gather}%
      \Phi_{\text{in}}^{(2)} = -\frac{Q}{4\pi\eps_0} \frac{\al^2/4}{\ab|\bs{x}-R\hat{\bs{z}}|}
    \end{gather}%
    となり,$\Phi_{\text{in}}^{(1)} + \Phi_{\text{in}}^{(2)}$が(c-1-a)の
    答えになっていることが確かめられる.$\Phi_{\text{out}}$については$\Phi^{(1)}$の$1/R$を$1/r$
    として考えれば良く,この結果も整合的である.
  \end{tcolorbox}
  (a)で考えたポテンシャルを図示すると図\ref{fig:3-2_normal}のようになる.
  \begin{figure}[htbp]%  
    \centering%  
    \includegraphics[width=0.8\linewidth]{py/3-2_normal.png}%  
    \caption{$\alpha = \pi/4$としてポテンシャルを描いた様子.
    ポテンシャルの単位は$Q/(4\pi\eps_0)$としており,球(図中の灰色の実線)
    の半径を$R = 1$としている.
    }%  
    \label{fig:3-2_normal}%  
  \end{figure}%

  \hrulefill\\
  \toi{3.3}
  \itemlabel{(a)}
  空間内の電荷分布は
  \begin{gather}%
    \rho_{\mr{c}}(\bs{x}) = \frac{a}{\sqrt{R^2-\rho^2}} H(R-\rho) \delta(z)
  \end{gather}%
  [ただし$H(\cdot)$はHeavisideのstep functionである]とかけるので,$a$を決定する.
  原点でのポテンシャルは
  \begin{gather}%
    V(0) = \frac{1}{4\pi \eps_0} \int \frac{1}{\rho'}\rho_{\mr{c}}(\bs{x'}) 
    \rho'\dl{\rho'} \dl{\phi'} \dl{z'}
    = \frac{\pi a}{4\eps_0}
  \end{gather}%
  と表されるので,$V(0) = V$として,電荷分布は
  \begin{gather}%
    \rho_{\mr{c}}(\bs{x}) = \frac{4\eps_0 V}{\pi} \frac{1}{\sqrt{R^2 - \rho^2}} 
    H(R-\rho) \delta(z)
  \end{gather}%
  である.
  これより,$z$軸上でのポテンシャルを計算することができて,
  \begin{align}%
    \Phi(z) &= \frac{1}{4\pi \eps_0} \int 
    \frac{\rho_{\mr{c}}(\bs{x'})}{\sqrt{(\rho')^2+z^2}} 
    \rho' \dl{\rho',\phi',z'}\notag\\
    &=\frac{2V}{\pi} \int_0^R \frac{\rho' \dl{\rho'}}
    {\sqrt{
      ((\rho')^2+z^2) (R^2-(\rho')^2)
    }}\notag\\
    &= \frac{2V}{\pi} \int_0^{\pi/2} 
    \frac{R\sin\theta\dl{\theta}}{\sqrt{z^2 + R^2\sin\theta}}\notag\\
    &= \frac{2V}{\pi} \arctan\ab(\frac{R}{|z|})
  \end{align}%
  と表すことができる[式\eqref{eq:3.676}を利用].

  $R / |z| \leq 1$とすると,
  \begin{gather}%
    \Phi(z) = \frac{2V}{\pi} \sum_{k=0}^\infty \frac{(-1)^k}{2k+1} \ab(\frac{R}{|z|})^{2k+1}
  \end{gather}%
  として級数展開することができるので,$R / r \leq 1$かつ$z \geq 0$のdisk上部では
  \begin{gather}%
    \Phi(\bs{x}) = 
    \frac{2V}{\pi}\sum_{l=0}^\infty \frac{(-1)^l}{2l+1} \ab(\frac{R}{r})^{2l+1}
    P_{2l}(\cos\theta)
  \end{gather}%
  と表される.$z \leq 0$のdisk%
  \footnote{diskとdiscは技術的な文脈では使い分けがあるらしい.
  discはCDやDVDなどの光学式メディアを指し,diskはフロッピーディスクやハードディスク
  などを指すらしい.}
  下部では$z=0$面に関してポテンシャルが対称である
  として,
  \begin{align}%  
    \Phi(\bs{x}) & = \frac{2V}{\pi}\sum_{l=0}^\infty \frac{(-1)^l}{2l+1} \ab(\frac{R}{r})^{2l+1}
    P_{2l}(\cos(\pi-\theta))\\
    &= 
    \frac{2V}{\pi}\sum_{l=0}^\infty \frac{(-1)^l}{2l+1} \ab(\frac{R}{r})^{2l+1}
    P_{2l}(\cos\theta)
  \end{align}%
  となる.

\itemlabel{(b)}
  一方$R / |z| \geq 1$とすると,
  $\arctan(x) + \arctan(1/x) = \pi/2$に注意して,
  \begin{align}%
    \Phi(z) &= \frac{2V}{\pi}\ab[\frac{\pi}{2}-\arctan\ab(\frac{|z|}{R})]\notag\\
    &= V - \frac{2V}{\pi} \sum_{l=0}^\infty \frac{(-1)^l}{2l+1}\ab(\frac{|z|}{R})^{2l+1}
  \end{align}%
  であるから,
  $z \gtrless 0$では
  \begin{align}%
    \Phi(\bs{x}) &=
    \begin{dcases}
      V - \frac{2V}{\pi} \sum_{l=0}^\infty \frac{(-1)^l}{2l+1} \ab(\frac rR)^{2l+1} P_{2l+1}(\cos\theta)\\
      V - \frac{2V}{\pi} \sum_{l=0}^\infty \frac{(-1)^l}{2l+1} \ab(\frac rR)^{2l+1} P_{2l+1}(\cos(\pi-\theta))
    \end{dcases}\notag\\
    &= V \mp \frac{2V}{\pi}
    \sum_{l=0}^\infty \frac{(-1)^l}{2l+1} \ab(\frac rR)^{2l+1} P_{2l+1}(\cos\theta)
  \end{align}
  となる.

\itemlabel{(c)}
  導体上の全電荷は
  \begin{gather}%
    Q \equiv \int_0^R 2\pi \rho \dl{\rho} 
    \frac{4\eps_0 V}{\pi} \frac{1}{\sqrt{R^2  -\rho^2}} = 8\eps_0 VR
  \end{gather}%
  であるから,capacitanceは$C = Q/V = 8\eps_0 R$
  である.
  \begin{figure}[H]%  
    \centering%  
    \includegraphics[width=0.8\linewidth]{py/3-3_normal_w_circle.png}%  
    \caption{$V=1.0$,$R = 1.0$として等ポテンシャル面を描画した図;
    ポテンシャルが$V$に保たれているdiskは$z$軸上の$-1 < x <1$の範囲である.
    また,図中の黄色破線は$r = R$となる球面を表しており,この面を境にして
    ポテンシャルの表式が変化している.}%  
    \label{fig:3-3_normal_w_circle}%  
  \end{figure}%

\hrulefill\\
\afterpage{\clearpage}
%---- problem 3.4 ---%
\toi{3.4}
\begin{wrapfigure}[12]{r}{0.3\linewidth} 
  \centering
  \includegraphics[width=\linewidth]{fig/Jackson3-4_1.pdf}\\
  \includegraphics[width=\linewidth]{fig/Jackson3-4_2.pdf}
\end{wrapfigure}%
\itemlabel{(a)}
  球内部でのポテンシャルは,原点での特異性に気をつけると,
  \begin{gather}%  
    \label{eq:3-4_eq1}
    \Phi(r,\theta,\phi) = \sum_{l=0}^\infty \sum_{m=-l}^{l} A_{lm}r^l Y_{lm}(\theta,\phi)
  \end{gather}%
  と書くことができる.
  境界条件は,
  \begin{multline}%
    \Phi(r=a,\theta,\phi) \equiv V(\phi)\\=
    \begin{dcases}%
      +V \qqtext{if} \frac{\pi}{n} \cdot 2j \leq \phi \leq \frac{\pi}{n}\cdot(2j+1)\\
      -V \qqtext{if} \frac{\pi}{n}\cdot(2j+1) \leq \phi \leq \frac{\pi}{n}\cdot (2j+2)
    \end{dcases}\\%
    \qqtext{for} j = 0, 1, \ldots, n-1
  \end{multline}%
  である.
  \begin{gather}%
    V(\phi) = \sum_{l} \sum_{m} A_{lm} a^l Y_{lm{(\theta,\phi)}}
  \end{gather}%
  に対して,両辺に$Y^*_{l'm'}(\theta,\phi)$をかけて,
  $\dl{\Omega} = \sin\theta\dl{\theta,\phi}$で積分を実行すると,
  球面調和関数の直交性より
  \begin{align}%
    A_{lm} &= \frac{1}{a^l} \int V(\phi) Y^*_{lm}(\theta,\phi) \dl{\Omega}\notag\\
    &= \frac{1}{a^l} \sqrt{\frac{2l+1}{4\pi}\frac{(l-m)!}{(l+m)!}}\int_0^\pi
    P_{l}^m(\cos\theta)\sin\theta\dl{\theta} \int_0^{2\pi} V(\phi) \e^{-\i m\phi}\dl{\phi}
  \end{align}%
  である.$\phi$での積分について,
  \begin{multline}
    \int_0^{2\pi} V(\phi) \e^{-\i m\phi} \dl{\phi} = V \sum_{j=0}^{n-1} \ab[
      \int_{\frac{\pi}{n}2j}^{\frac{\pi}{n}(2j+1)} \e^{-\i m\phi}\dl{\phi}
       - \int_{\frac{\pi}{n}(2j+1)}^{\frac{\pi}{n}(2j+2)} \e^{-\i m\phi}\dl{\phi}
    ]
  \end{multline}
  となるが, $m = 0$のとき,
  \begin{gather}%
    \int_0^{2\pi} V(\phi)\e^{-\i m\phi} \dl{\phi} =  0
  \end{gather}%
  であり,$m \neq 0$のときは
  \begin{gather}%
    \int_0^{2\pi} V(\phi) \e^{-\i m\phi} \dl{\phi} =
      - \frac{\i V}{m} \ab(1-\e^{-\i \frac{m}{n}\pi})^2 \sum_{j=0}^{n-1}\ab(\e^{-\i \frac{2m}{n}\pi})^j
  \end{gather}%
  となるが,$\exp(-\i (m/n)\pi) = 1$つまり$m/(2n)$が整数となるときはゼロとなる.
  また,
  \begin{gather}%
    \sum_{j=0}^{n-1} \ab(\e^{-\i\frac{2m}{n}\pi})^j 
    = \frac{1 - \e^{-\i 2m\pi}}{1-\e^{\i \frac{2m}{n}\pi}}
  \end{gather}%
  となるので,$m/n$が整数となるときには右辺が$0 / 0$の形になるので,non-zeroとなることが
  期待される.一方,それ以外の場合では$m$が整数であることより分子がゼロになるので,
  この和はゼロである.
  以上をまとめると,$m / n = \pm1, \pm3, \ldots$の場合のみが
  non-zeroの結果を与えることがわかる.実際,このときは
  \begin{gather}%  
    \int_0^{2\pi} V(\phi) \e^{-\i m\phi} \dl{\phi} =
    -\frac{\i V}{m}\ab(1-(-1))^2 \sum_{j=0}^{n-1} 1 = -4\i V \frac{n}{m}
  \end{gather}%
  である.
  以上の結果を踏まえると,球内部でのポテンシャルを\eqref{eq:3-4_eq1}のように展開したとき,
  係数$A_{lm}$は
  \begin{gather}%
    A_{lm} = 
    \begin{dcases}%
      \begin{multlined}
        -\frac{4\i V}{a^l}\frac{n}{m} \sqrt{\frac{2l+1}{4\pi}\frac{(l-m)!}{(l+m)!}} 
        \int_{-1}^1 P_l^m (\cos\theta)\dl{(\cos\theta)} \\
        \qqtext{if} m = \pm n, \pm 3n, \pm 5n, \ldots
      \end{multlined}\\
      0 \qqtext{otherwise}
    \end{dcases}%
  \end{gather}%

\itemlabel{(b)}
  $n = 1$として係数$A_{lm}$を$l = 3$まで計算する.$m = \pm 1, \pm 3, \ldots$に対して
  $A_{lm} \neq 0$であることに注意する.
  \begin{gather*}%
    A_{1,\pm 1} = \frac{\i V}{a}\sqrt{\frac{3\pi}{2}},\qq
    A_{2,\pm 1} = 0\\
    A_{3,\pm 3} = \frac{\i V}{a^3} \sqrt{\frac{35\pi}{256}},\qq
    A_{3,\pm 1} = \frac{\i V}{a^3} \sqrt{\frac{21\pi}{256}}
  \end{gather*}%
  と計算できるので,ポテンシャルは
  \begin{multline}%
    \Phi(r,\theta,\phi) = V\left[
      \frac{3r}{2a} \sin\theta\sin\phi\right.  \\
      \left.
      +
      \ab(\frac{r}{a})^3\ab\{
        \frac{35}{64}\sin^3\theta\sin(3\phi) + 
        \frac{21}{64}\sin\theta\ab(5\cos^2\theta-1)\sin\phi
      \} + \cdots
    \right]
  \end{multline}%
  とかける.ここで,$\cos\theta' = \sin\theta \sin\phi$とおけば,
  \begin{gather}%
    P_1(\cos\theta') = \sin\theta\sin\phi\\
    P_3(\cos\theta') = -\frac{1}{8}\ab[5\sin^3\theta\sin(3\phi) 
    + 3\sin\theta(5\cos^2-1)\sin\phi]
  \end{gather}%
  であるから,
  \begin{gather}%
    \Phi(r,\theta') = V\ab[\frac{3}{2} \frac ra P_1(\cos\theta') - 
    \frac{7}{8}\ab(\frac ra)^3P_3(\cos\theta') + \cdots]
  \end{gather}%
  となり,これは,教科書本文中の式(3.36)と一致していることがわかる.
  図\ref{fig:3-4_n3}には$n = 3$の場合の図を描いた.
  \begin{figure}[H]%  
    \centering%  
    \includegraphics[width=0.8\linewidth]{py/3-4_n3.png}%  
    \caption{問題文と同じセットアップを行って,$n = 3$の場合について,$l = 20$の
    項まででポテンシャルを描いた図.球($z = 0$面で切断したときの円)の境界
    の近くでは有限項で打ち切った事による誤差が見られるが,概ね期待されるポテンシャルの
    形を得られていることがわかる.}%  
    \label{fig:3-4_n3}%  
  \end{figure}%

  \hrulefill\\
  \toi{3.5}
  %--- problem 3.5 ---%
  本文中の式(3.70)より,
  \begin{gather}%
    \frac{1}{\ab|\bs{x}-\bs{x}'|} = 4\pi \sum_{l=0}^\infty \sum_{m=-l}^l \frac{1}{2l+1}
    \frac{r^l}{a^{l+1}} Y^*_{lm}(\theta',\phi') Y_{lm}(\theta,\phi)
  \end{gather}%
  である(ただし$|\bs{x}| = r < a = |\bs{x}'|$とした).このとき,
  \begin{gather*}%  
    |\bs{x} - \bs{x}'| = \ab(r^2 + a^2 - 2ar\cos\gamma)^{1/2}
  \end{gather*}%
  である($\gamma$は$\bs{x}, \bs{x}'$のなす角)から,
  \begin{gather}%
    \label{eq:3-5_eq1}
    \frac{1}{\ab(r^2 + a^2 - 2ar \cos\gamma)^{1/2}} = 4\pi\sum_{l,m} \frac{1}{2l+1}
    \frac{r^l}{a^{l+1}} Y^*_{lm}(\theta',\phi') Y_{lm}(\theta,\phi)
  \end{gather}%
  とかける.ここで,\eqref{eq:3-5_eq1}の両辺を$r$について微分すると,
  \begin{gather}%
    \label{eq:3-5_eq2}
    \frac{ar\cos\gamma - r^2}{\ab(r^2+a^2-2ar\cos\gamma)^{3/2}} = 4\pi \sum_{l,m}
    \frac{l}{2l+1} \frac{r^l}{a^{l+1}} Y^*_{lm}(\theta',\phi') Y_{lm}(\theta,\phi)
  \end{gather}%
  が得られ,同様に\eqref{eq:3-5_eq1}の両辺を$a$について微分すると,
  \begin{gather}%
    \label{eq:3-5_eq3}
    \frac{ar\cos\gamma - a^2}{\ab(r^2 + a^2 -2ar\cos\gamma)^{3/2}} = 
    -4\pi\sum_{l,m}\frac{l+1}{2l+1}\frac{r^l}{a^{l+1}}
    Y^*_{l,m}(\theta',\phi') Y_{lm}(\theta,\phi)
  \end{gather}%
  が得られる.ここで,\eqref{eq:3-5_eq2}と\eqref{eq:3-5_eq3}の差を考えることで,
  \begin{gather}%
    \frac{a(a^2 - r^2)}{\ab(r^2 + a^2 - 2ar\cos\gamma)^{3/2}}
    = 4\pi \sum_{l,m} \ab(\frac r a)^l Y^*_{lm}(\theta',\phi') Y_{lm}(\theta,\phi)
  \end{gather}%
  となる.この両辺に$V(\theta',\phi')$をかけて,$\dl{\Omega'}$で積分すると,
  \begin{gather}%
    \frac{a(a^2 - r^2)}{4\pi} \int  
    \frac{V(\theta',\phi')\dl{\Omega'}}{\ab(r^2 + a^2 - 2ar\cos\gamma)^{1/2}} = 
    \sum_{l,m} A_{lm} \ab(\frac ra)^l Y_{lm}(\theta,\phi)
  \end{gather}%
  となる.ただし,
  \begin{gather}%
    A_{lm} = \int \dl{\Omega'} V(\theta',\phi') Y^*_{lm}(\theta',\phi')
  \end{gather}%
  である.
  
  \hrulefill\\
  \toi{3.6}
  %--- problem 3.6 ---%
\itemlabel{(a)}
  点電荷によるポテンシャルは次のようにかける:
  \begin{gather}%
    \Phi(\bs{x}) = \frac{q}{4\pi \eps_0}
    \ab[\frac{1}{\ab|\bs{x} - a\hat{\bs{z}}|}
    - \frac{1}{\ab|\bs{x} + a \hat{\bs{z}}|}].
  \end{gather}%
  本文中の式(3.70)を利用して書き直すと,
  \begin{gather}%
    \Phi(\bs{x}) = \frac{q}{\eps_0} \sum_{l,m} \frac{1}{2l+1} \frac{r_<^l}{r_>^{l+1}}
    Y_{lm}(\theta,\phi) \ab[Y^*_{lm}(0,0) - Y^*_{lm}(\pi,0)]
  \end{gather}%
ここで,$Y^*_{lm}(0,0)-Y^*_{lm}(\pi,0) \neq 0$となるのは$m = 0$で$l$が奇数のときのみである
ことに注意すると,
\begin{gather}%
  \Phi(\bs{x}) = \frac{q}{\eps_0} \sum_{l=1,3,\ldots}^{\infty} \frac{1}{2l+1} \frac{r_<^l}{r_>^{l+1}} \sqrt{\frac{2l+1}{4\pi}}P_l(\cos\theta) 
  \cdot \sqrt{\frac{2l+1}{4\pi}}\cdot2
\end{gather}%






