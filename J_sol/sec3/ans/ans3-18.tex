\begin{bx1}
  問題3.12の状況設定に対して,円盤に平行になるように
  $L$だけ離れたところにポテンシャルをゼロに保った導体平板を挿入することを考える.
  配置を明確にするために,接地させた平板を$z = 0$とし,
  $z$軸上に中心を持つその他の円盤や平板は$z = L$であるとする.
  \begin{enumerate}[(a)]%  
    \item
      平板間のポテンシャルが円筒座標系$(z, \rho, \phi)$を用いて,次のように
      表されることを示せ:
      \begin{gather}%
        \Phi(z, \rho) = V \int_0^\infty \dl{\lambda} J_1(\lambda) J_0 \ab(\frac{\lambda \rho}{a}) \frac{\sinh(\lambda z / a)}{\sinh(\lambda L/a)}.
      \end{gather}%
    \item 
      $z, \rho, L$を固定して$a \to \infty$とするときに,(a)の部分の解が期待される結果に
      帰着されることを示せ.この結果を$a^{-1}$のべき級数展開における最低次の項とみなして,
      $a$が$\rho$や$L$に比べて大きいが無限ではない場合,この最低次の表式に対する
      補正項について考察せよ.このとき,何か問題が生じるか?また,
      その補正項の具体的な大きさを見積もることは可能か?
    \item 
      $(L-z), a, \rho$を固定して$L \to \infty$の極限を考え,問題3.12の結果を再現する
      ことを示せ.また,$L \gg a$であるが$L \to \infty$ではない場合の補正項は
      どのようになるか.
  \end{enumerate}%
\end{bx1}
\clearpage
