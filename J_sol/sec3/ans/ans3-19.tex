\begin{bx1}
  ポテンシャルがゼロに保たれた,2つの無限に広がる平行平板間に存在する点電荷$q$を考える.
  円筒座標系で平板が$z = 0$と$z = L$に存在し,点電荷は$z$軸上の$z = z_0 \, (0 < z_0 < L)$
  に存在するとする.問題3.18と比較して,問題1.12のグリーンの相反定理
  (Green's reciprocation theorem)を用いよ.
  \begin{enumerate}[(a)]%  
    \item
      $z = L$の平板において,$z$軸を中心とする半径$a$の円内部に誘導される電荷が次の式で
      与えられることを示せ:
      \begin{gather}%
        Q_L(a) = -\frac{q}{V} \Phi(z_0, 0).
      \end{gather}%
      ここで,$\Phi(z_0, 0)$は問題3.18において$z = z_0, \rho = 0$で計算した
      ポテンシャルである.また,上側の平面に誘導される総電荷を求めよ.この結果を
      (方法とその結果において)問題1.13と比較せよ.
    \item 
      上側の平面に誘導された電荷密度は次のように書けることを示せ:
      \begin{gather}%
        \sigma(\rho) = -\frac{q}{2\pi} \int_0^\infty \dl{k} \frac{\sinh(kz_0)}{\sinh(kL)} kJ_0(k\rho).
      \end{gather}%
      また,この積分は変形ベッセル関数$K_0(n\pi \rho/L)$を含む無限級数とし
      て表現することもでき
      \footnote{
        \cite{gradshteyn2014}の式(6.666):
        \begin{gather}%
          \int_0^\infty  x^{\nu+1} \frac{\sinh(\alpha x)}{\sinh(\pi x)} J_\nu(\beta x) \dl{x}
          = \frac{2}{\pi} \sum_{n=1}^\infty (-1)^{n-1} n^{\nu+1} \sin(\pi\alpha) K_\nu(n \beta) 
          \qq [|\Re(\alpha)| < \pi,\, \Re(\nu) > -1]
        \end{gather}%
      },動径方向の距離が大きいときに誘導された電荷密度が
      $\rho^{-1/2}\e^{-\pi\rho/L}$で減少することがわかる.
    \item 
      $\rho = 0$における電荷密度が,次の級数として書けることを示せ:
      \begin{gather}
        \sigma(0) = -\frac{q}{2\pi L^2}\sum_{\substack{n>0\\n:\,\text{odd}}} \ab[
          \ab(n-\frac{z_o}{L})^{-2} - \ab(n+ \frac{z_0}{L})^{-2}
        ].
      \end{gather}
  \end{enumerate}%

\end{bx1}
\clearpage
