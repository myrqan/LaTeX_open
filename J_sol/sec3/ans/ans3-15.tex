\begin{bx1}
  外部回路に接続された電源の「球形の牛(spherical cow)」モデルについて考える.
  半径$a$,導電率$\sigma$の球が導電率$\sigma'$の一様な媒質の中に埋め込まれている
  とする.球の内部では,電荷担体(charge carrier)に対して一様な(化学的な)力が$z$方向に
  働き,その大きさはオームの法則における実効的な電場として$F$である.
  定常状態では,電場は球の内外に存在し,表面電荷は球表面に分布している.
  \begin{enumerate}[(a)]%  
    \item
      ($F$に加えての)電場の大きさと,空間内の任意の点における電流密度を求めよ.
      また,表面電荷密度を決定し,球の電気双極子モーメントが
      $p = 4\pi \eps_0\sigma a^3 F/ (\sigma + 2\sigma')$であることを
      示せ.
    \item 
      球の上半球から流れ出る電流が次の式で表されることを示せ:
      \begin{gather}%
        I = \frac{2\sigma\sigma'}{\sigma + 2\sigma'} \cdot \pi a^2 F.
      \end{gather}%
      球の外部で損失する総電力を計算せよ.集中定数回路\footnote{
        回路素子が有限の個数で表現される回路のこと(簡単に言うと,回路素子以外の
        導線などの影響を無視した回路のこと).回路素子間のインピーダンスが
        十分小さく無視することが可能である.対になるものとして,分布定数回路がある.
      }の関係式$P = I^2 R_{\mr{e}} = IV_{\mr{e}}$を用いて,有効外部抵抗$R_{\mr{e}}$%
      と有効外部電圧$V_{\mr{e}}$を求めよ.
    \item
      球内部で損失する電力を求め,それより有効内部抵抗$R_{\mr{i}}$と有効内部電圧$V_{\mr{i}}$
      を求めよ.
    \item
      関係式$V_{\mr{t}} = (R_{\mr{e}} + R_{\mr{i}})I$を用いて総電圧を定義し,
      $V_{\mr{t}} = 4aF / 3$となること,および
      $V_{\mr{e}}+V_{\mr{i}} = V_{\mr{t}} $となること
      を示せ.また,$IV_{\mr{t}}$が「化学」的な力から供給される電力であることを示せ.
  \end{enumerate}%
\end{bx1}

\clearpage
