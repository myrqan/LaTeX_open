\begin{bx1}
  \begin{wrapstuff}[type=figure,r,lines=7,width=0.23\linewidth]
    \includegraphics[width=\linewidth]{fig/Jackson3-1.pdf}%  
  \end{wrapstuff}
  共通中心を持つ2つの球(それぞれの半径は$a, b$で与えられ,$b>a$を満たす)があり,それぞれは
  共通の水平板によって2つの半球に分割されている.内部の球の上側と外側の球の下側のポテンシ
  ャルは$V$に固定され,その他の半球のポテンシャルはゼロに固定されている.

  このとき,$a \leq r \leq b$の領域におけるポテンシャルをルジャンドル多項式による展開として
  決定せよ.このとき,少なくとも$l = 4$の項までを含めよ.ここで得られた結果と,
  よく知られた結果である$b \to \infty$や$a \to 0$の場合と比較せよ.
\end{bx1}
半径$a, b$の球の間の領域のポテンシャルを考える.Legendre多項式による展開
\begin{gather}%
  \Phi(r, \theta) = \sum_{l=0}^\infty \ab(A_l r^l + B_l r^{-(l+1)}) P_l (\cos\theta)
\end{gather}
を以下の境界条件で考える:
\begin{gather}%  
  \label{eq:prob3-1_bc1}
  \Phi(a,\theta) =
  \begin{dcases}%
    V \qqtext{if} 0 \leq \theta \leq \frac{\pi}{2}\\
    0 \qqtext{if} \frac{\pi}{2} \leq \theta \leq \pi
  \end{dcases},\\
  \label{eq:prob3-1_bc2}
  \Phi(b,\theta) =
  \begin{dcases}%
    0 \qqtext{if} 0 \leq \theta \leq \frac{\pi}{2}\\
    V \qqtext{if} \frac{\pi}{2} \leq \theta \leq \pi
  \end{dcases}% 
\end{gather}%
\eqref{eq:prob3-1_bc1}について,$P_{l'}(\cos\theta)\dl{(\cos\theta)}$をかけて$\theta\in[0,\pi]$
で積分をすると,
\begin{gather}%
  \int_{\cos\theta=1}^{\cos\theta=-1} \sum_{l=0}^\infty\ab(A_la^l + B_la^{-(l+1)})
  P_l(\cos\theta) P_{l'}(\cos\theta) \dl{(\cos\theta)} 
  =  V \int_{\cos\theta=1}^{\cos\theta=0} P_{l'}(\cos\theta) \dl{(\cos\theta)}
\end{gather}
である.左辺の積分変数を$x$で書くと,
\begin{gather}%
  \text{(L.H.S.)} = \int_{1}^{-1} \sum_{l=0}^\infty\ab(A_l a^l + B_l a^{-(l+1)})
  P_l(x) P_{l'}(x) \dl{x}
\end{gather}
であり,Legendre多項式に関する直交関係\eqref{eq:legendre_orthogonal}より,($l,l'$を入れ替えて)
\begin{gather}%
  \label{eq:prob3-1_a}
  A_l a^l + B_l a^{-(l+1)} = \frac{V(2l+1)}{2} \int_0^1 \dl{x} P_l(x)
\end{gather}%
を得る.
同様に考えて,\eqref{eq:prob3-1_bc2}について,
\begin{gather}%
  \label{eq:prob3-1_b}
  A_l b^l + B_l b^{-(l+1)} =(-1)^l \frac{V(2l+1)}{2} \int_0^{1} \dl{x} P_l(x)
\end{gather}%
が成り立つ.
ただし,$P_l(-x) = (-1)^l P_l(x)$に注意せよ.

式\eqref{eq:prob3-1_a}と\eqref{eq:prob3-1_b}より,係数$A_l, B_l$を計算することができて,
\begin{gather}%
  A_l = \frac{(-1)^lb^{l+1}-a^{l+1}}{b^{2l+1}-a^{2l+1}}
  \frac{V(2l+1)}{2} \int_0^1 \dl{x} P_l(x),\\
  B_l = \frac{b^l-(-1)^la^l}{b^{2l+1}-a^{2l+1}}a^{l+1}b^{l+1}
  \frac{V(2l+1)}{2}\int_0^1 \dl{x} P_l(x)
\end{gather}%
がわかる.
したがって,ポテンシャルの正確な形は
\begin{gather}
  \Phi(r,\theta) =
  V \sum_{l=0}^\infty \ab(
  \frac{(-1)^lb^{l+1}-a^{l+1}}{b^{2l+1}-a^{2l+1}} r^l +
  \frac{b^l-(-1)^la^l}{b^{2l+1}-a^{2l+1}}a^{l+1}b^{l+1} r^{-(l+1)}
  )
  \times\frac{2l+1}{2}\ab(\int_0^1 \dl{x} P_l(x))
  P_l(\cos\theta)
\end{gather}

である.
具体的に$l = 4$の項までを書き下せば,
\begin{gather}
  \Phi(r,\theta) = V \left[
    \frac{1}{2} + \frac{3}{4}\frac{-(a^2+b^2)r+a^2b^2(a+b)r^{-2}}{b^3-a^3} P_1(\cos\theta)\right.
    \left.
    -\frac{7}{16}\frac{-(a^4+b^4)r^3 + a^4b^4(a^3+b^3)r^{-4}}{b^7-a^7}P_3(\cos\theta) +\cdots
    \right]
\end{gather}
となる.

$b \to \infty$の極限では$a/b \to 0, r/b\to 0$として,
\begin{align}%
  \Phi(r,\theta) 
  &= V
  \sum_{l=0}^\infty \frac{2l+1}{2}\ab(\frac{a}{r})^{l+1}
  \ab(\int_0^1\dl{x} P_l(x)) P_l(\cos\theta)\notag\\
  &= V\ab[
    \frac{1}{2} + \sum_{k=0}^\infty \frac{(-1)^k(4k+3)(2k-1)!!}{2^{k+2}(k+1)!} 
    \ab(\frac ar)^{2k+2} P_{2k+1}(\cos\theta)
  ]\\
  &= V\ab[
    \frac{1}{2} + \frac{3}{4}\ab(\frac{a}{r})^2 P_1(\cos\theta) -
    \frac{7}{16}\ab(\frac{a}{r})^4 P_3(\cos\theta) + \cdots
  ]
\end{align}%
である.
ただし,ここでは$(-1)!! = 1$と約束する.
これは半径$a$の導体球について,
北半球のポテンシャルが$V$,南半球が接地されており,さらに
無限遠でのポテンシャルの境界条件が$V/2$となる系において,導体球の外部領域に作る
ポテンシャルに等しいことがわかる.

$a \to 0$の極限では
\begin{align}%
  \Phi(r,\theta) &= V \sum_{l=0}^\infty (-1)^l \frac{2l+1}{2} \ab(\frac rb)^{l} 
  \ab(\int_0^1\dl{x} P_l(x)) P_l(\cos\theta)\notag\\
  &= V\ab[\frac 12 + \sum_{k=0}^\infty \frac{(-1)^{k+1}(4k+3)(2k-1)!!}{2^{k+2}(k+1)!}
  \ab(\frac rb)^{2k+1} P_{2k+1} (\cos\theta)]\\
  &= V\ab[\frac 12 - \frac 34 \ab(\frac rb) P_1(\cos\theta) +
  \frac{7}{16}\ab(\frac rb)^3P_3(\cos\theta)]
\end{align}%
である.
これは,半径$b$の導体球について,
北半球が接地されていて,南半球のポテンシャルが$V$となるような系における,
球内部のポテンシャルに等しいことがわかる.

それぞれの場合について,ポテンシャルを描く
と図\ref{fig:3-1_lima},\ref{fig:3-1_limb},\ref{fig:3-1_all}のようになる\footnote{
  描画のときに使用したコードは\texttt{.ipynb}形式で保存してある.
}.
\begin{figure}[htbp]%  
  \centering%  
  \begin{minipage}{0.30\linewidth}
    \centering
    \includegraphics[width=\linewidth]{py/3-1_lima.png}%  
    \caption{$a \to 0$としたときのポテンシャル}%  
    \label{fig:3-1_lima}%  
  \end{minipage}
  \begin{minipage}{0.30\linewidth}
    \centering
    \includegraphics[width=\linewidth]{py/3-1_limb.png}%  
    \caption{$b\to\infty$としたときのポテンシャル}%  
    \label{fig:3-1_limb}%  
  \end{minipage}
  \begin{minipage}{0.30\linewidth}
    \centering
    \includegraphics[width=\linewidth]{py/3-1_all.png}
    \caption{一般の場合についてのポテンシャル}
    \label{fig:3-1_all}
  \end{minipage}
\end{figure}%

\clearpage
