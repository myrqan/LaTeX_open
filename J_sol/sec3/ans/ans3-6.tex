\begin{bx1}
  2つの点電荷$+q,-q$が$z$軸上の点$z = a$と$z = -a$にそれぞれ存在している.
  \begin{enumerate}[(a)]%  
    \item 
      球面調和関数と$r$の冪による級数展開の形で
      $r<a$および$r>a$におけるポテンシャルを決定せよ.
    \item 
      積$qa \equiv p / 2$を一定に保ちながら$a \to 0$としたときの$r \neq 0$におけるポテンシャル
      を決定せよ.これは,定義より$z$軸に沿った双極子およびそのポテンシャルを表している.
    \item (b)で考えた双極子が\emph{接地された},原点を中心とする半径$b$の球殻で
      取り囲まれているとする.このとき,線形の重ね合わせにより,
      球殻内部におけるポテンシャルを決定せよ.
  \end{enumerate}%
\end{bx1}

\itemlabel{(a)}
  点電荷によるポテンシャルは次のようにかける:
  \begin{gather}%
    \Phi(\bs{x}) = \frac{q}{4\pi \eps_0}
    \ab[\frac{1}{\ab|\bs{x} - a\hat{\bs{z}}|}
    - \frac{1}{\ab|\bs{x} + a \hat{\bs{z}}|}].
  \end{gather}%
  本文中の式(3.70)を利用して書き直すと,
  \begin{gather}%
    \Phi(\bs{x}) = \frac{q}{\eps_0} \sum_{l,m} \frac{1}{2l+1} \frac{r_<^l}{r_>^{l+1}}
    Y_{lm}(\theta,\phi) \ab[Y^*_{lm}(0,0) - Y^*_{lm}(\pi,0)]
  \end{gather}%
  となるが,ここで,$Y^*_{lm}(0,0)-Y^*_{lm}(\pi,0) \neq 0$となるのは$m = 0$で
  $l$が奇数のときのみであることに注意すると,
  \begin{align}%
    \Phi(\bs{x}) &= \frac{q}{\eps_0} \sum_{l=1,3,\ldots}^{\infty} \frac{1}{2l+1} \frac{r_<^l}{r_>^{l+1}} \sqrt{\frac{2l+1}{4\pi}}P_l(\cos\theta) 
    \cdot \sqrt{\frac{2l+1}{4\pi}}\cdot 2\notag\\
    &= \frac{q}{2\pi\eps_0} \sum_{l=1,3,\ldots}^\infty \frac{r_<^l}{r_>^{l+1}} P_l(\cos\theta)
  \end{align}%
  と書き直される.ただし,$r_>, r_<$はそれぞれ$r, a$の大きい方,小さい方である.

\itemlabel{(b)}
  $a\to0$の極限を考えるので,球外部のポテンシャルを考えることにする
  [$r_< = a$および$r_> = r$
  である].
  このとき,
  \begin{gather}%
    \Phi(\bs{x}) = \frac{q}{2\pi\eps_0}\sum_{l=1,3,\ldots}\frac{a^l}{r^{l+1}} P_l(\cos\theta)
  \end{gather}%
  であり,$qa \equiv p/2$を一定にして$a \to 0$の極限を考えると,
  \begin{gather}%
    \Phi(\bs{x}) = \frac{p}{4\pi\eps_0} \frac{1}{r^2}\cos\theta
  \end{gather}%
  となる.

\itemlabel{(c)}
  dipoleによるポテンシャルとの重ね合わせとして
  \begin{gather}%
    \Phi(\bs{x}) = \frac{p}{4\pi\eps_0} \frac{1}{r^2}\cos\theta + \sum_{l=0}^\infty A_l r^l P_l(\cos\theta)
  \end{gather}%
  と書く.$r = b$(grounded sphere)では$\theta$の値によらず$\Phi = 0$となるので
  \begin{gather}%
    \frac{p}{4\pi\eps_0} \frac{1}{b^2} \cos\theta + \sum_l A_l b^l P_l(\cos\theta) = 0
  \end{gather}%
  である.
  $\cos\theta$についての恒等式と見ると,$P_l$の直交性より
  $A_1 =-p/(4\pi\eps_0 b^3)$で$l \neq 1$に対して$ A_l = 0 $となる.
  したがって,ポテンシャルは
  \begin{gather}%
    \Phi(\bs{x}) = \frac{p}{4\pi \eps_0} \ab(\frac{1}{r^2} - \frac{r}{b^3}) \cos\theta
  \end{gather}%
  と表される.

\clearpage
