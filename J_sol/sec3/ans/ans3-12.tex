\begin{bx1}
  無限に続く,薄い平板状の導体が,半径$a$の円状の孔を持っている.
  平板と同じ物質でできた,その孔よりもわずかに小さい,薄くて平らな円盤が
  平板とわずかな隙間をあけて存在している.
  円盤のポテンシャルは$V$に保たれ,平板のポテンシャルはゼロに保たれているとする.
  \begin{enumerate}[(a)]%  
    \item
      適切な円筒座標系を用いることにより,平板よりも上に存在する任意の点に対して,
      その点におけるポテンシャルをベッセル関数を含む積分表示を用いて表せ.
    \item
      円盤の中心から,平板に垂直に$z$だけ上空の点におけるポテンシャルが
      次の形で表されることを示せ:
      \begin{gather}%
        \Phi_0(z) = V \ab(1- \frac{z}{\sqrt{a^2 + z^2}}).
      \end{gather}%
    \item 
      円盤の縁から,平板に垂直に$z$だけ上空の点におけるポテンシャルが
      次の形で表されることを示せ:
      \begin{gather}%
        \Phi_a(z) = \frac{V}{2}\ab[1 - \frac{kz}{\pi a}K(k)].
      \end{gather}%
      ただし,$k = 2a/\sqrt{z^2+4a^2}$であり,$K(k)$は第一種完全楕円積分(complete elliptic integral of the first kind)である.
  \end{enumerate}
\end{bx1}
\clearpage
