\begin{bx1}
  接地された球内の直径に沿った,一様に帯電しているワイヤーによるポテンシャルの解として
  得られる(3.136)には混乱させるような側面がある.ワイヤーに十分近い(すなわち,
  $\rho = r\sin\theta \ll b$をみたす)ときにはポテンシャルは一様に帯電したワイヤー
  のもつポテンシャルに等しくなければならない.すなわち,$\Phi = (Q/4\pi\eps_0b)\log(b/\rho) + \Phi_0$の形をもつ必要がある.解(3.136)では明示的な形でこのふるまいをしない.
  \begin{enumerate}[(a)]%  
    \item Legendreの微分方程式(3.10)および,いくつかの部分積分を実行することで
      $\log(\csc\theta)$は球面調和関数によって適切な展開を実行することができ,
      (3.136)が次の形でも書き表されることを示せ:
      \begin{gather}%
        \Phi(\bs{x})  = \frac{Q}{4\pi\eps_0 b}\ab[\log\ab(\frac{2b}{r\sin\theta}) - 1 - 
        \sum_{j=1}^\infty\frac{4j+1}{2j(2j+1)} \ab(\frac rb)^{2j} P_{2j}(\cos\theta)
        ]
      \end{gather}%
      この表式では,ワイヤーの近くでの振る舞いが明らかである.また,定数項
      $\Phi_0 = -Q/(4\pi \eps_0 b)$の解釈を行え.なお,この表式では任意の$r / b < 1$
      に対して,すべての角度でLegendre多項式は急速に収束する.
    \item (3.38)の展開式を用いて
      \begin{gather}%
        \frac{1}{2} \ab(\frac{1}{\sin\theta/2} + \frac{1}{\cos\theta/2}) = 2 \sum_{j=0}^\infty P_{2j}(\cos\theta)
      \end{gather}%
      を証明せよ.さらに,球内部表面における面電荷密度の表式(3.137)が次のようにも
      表されることを示せ:
      \begin{gather}%
        \sigma(\theta) = - \frac{Q}{4\pi b^2} \ab[\frac{1}{2}\ab(\frac{1}{\sin\theta/2} + \frac{1}{\cos\theta/2}) - \sum_{j=0}^\infty \frac{1}{2j+1}P_{2j}(\cos\theta)].
      \end{gather}%
      この表式では$\theta = 0, \pi$における(可積分な)特異性が明示的に示される.また,
      級数は$\theta \to 0$で$\log(1/\theta)$の補正項を与える.
  \end{enumerate}%
  \tcblower
  なお,(3.136)と(3.137)は以下のとおりであった.
  \begin{gather}%
    \Phi(\bs{x}) = \frac{Q}{4\pi\eps_0 b}\ab[\log \ab(\frac br) +\sum_{j=1}^\infty \frac{4j+1}{2j(2j+1)} \ab\{1- \ab(\frac rb)^{2j}\}P_{2j}(\cos\theta)] \tag{J - 3.136}\\
    \sigma(\theta) = -\frac{Q}{4\pi b^2}\ab[1 + \sum_{j=0}^\infty \frac{4j+1}{2j+1}P_{2j}(\cos\theta)] \tag{J - 3.137}
  \end{gather}%

\end{bx1}

\itemlabel{(a)}
  $\log(\csc\theta) = \log(1/\sin\theta)$をLegendre多項式で展開して,
  \begin{gather}%
    \log\ab(\frac{1}{\sin\theta}) = \sum_{l=0}^\infty A_l P_l(\cos\theta)
  \end{gather}%
   のように表すことができるとする.このとき,係数$A_l$は
   \begin{gather}%
     A_l = \frac{2l+1}{2} \int_0^\pi \log\ab(\frac{1}{\sin\theta}) P_l(\cos\theta) \sin\theta \dl{\theta}
   \end{gather}%
   と表される.$\cos\theta = x$と置き換えて
   \begin{gather}%
     A_l = \frac{2l+1}{2} \int_{-1}^{1} \dl{x} \log\ab(\frac{1}{\sqrt{1-x^2}}) P_l(x)
   \end{gather}%
   計算を進める.

   $l = 0$の場合は$P_0(x) = 1$で,\eqref{eq:2.736}を用いて計算ができて
   \begin{gather}%
     A_0 = \frac{1}{2} \int_{-1}^1 \dl{x} \log\frac{1}{\sqrt{1-x^2}}  = 1 - \log 2
   \end{gather}%
   となる.また,$l \geq 1$のときについてはLegendreの微分方程式
   \eqref{eq:legedre_diffeq}の両辺に$\log(1/\sqrt{1-x^2})$をかけて$[-1,1]$で積分することで,
   \begin{gather}%
     A_l = -\frac{2l+1}{2l(l+1)} \int_{-1}^1 \diff**{x}{\ab[(1-x^2)\diff{P_l}{x}]} \log\ab(\frac{1}{\sqrt{1-x^2}}) \dl{x} = \frac{2l+1}{2l(l+1)}\ab[1 + (-1)^l]
   \end{gather}%
   を得る.したがって,$l=2j\,(j=1, 2, \ldots)$の場合のみがnon-zeroとなる.したがって,
   \begin{gather}%
     \log\ab(\frac{1}{\sin\theta}) = 1 - \log 2 + \sum_{j=1}^\infty \frac{4j+1}{2j(2j+1)} P_{2j}(\cos\theta)
   \end{gather}%
   と級数展開することができる.
   これを用いて,(3.136)式を表示すると,
   \begin{gather}%
     \Phi(\bs{x}) = \frac{Q}{4\pi\eps_0 b}\ab[\log\ab(\frac{2b}{r\sin\theta}) - 1 - \sum_{j =1}^\infty \frac{4j+1}{2j(2j+1)}\ab(\frac rb)^{2j} P_{2j}(\cos\theta)]
   \end{gather}%
   となる.
   また,定数項$\Phi_0 = -Q/(4\pi \eps_0 b)$は導体球の球面上に誘導された電荷
   によるポテンシャルを表している.

\itemlabel{(b)}
  まず,(3.38)式は,$\gamma$を$\bs{x}, \bs{x'}$のなす角として
  \begin{gather}%
    \frac{1}{|\bs{x}-\bs{x}'|} = \sum_{l=0}^\infty \frac{r_<^l}{r_>^{l+1}}P_l(\cos\gamma)
  \end{gather}%
  という式であった.ここで,$\gamma = \theta$,$|\bs{x}| = |\bs{x}'| = 1$
  とすると,
  \begin{gather}%
    \label{eq:pr3.8.1}
    \frac{1}{2\cos\theta/2} = \sum_{l=0}^\infty P_l(\cos\theta)
  \end{gather}%
  が導かれる.$\gamma = \pi -\theta$とすると,
  \begin{gather}%
    \label{eq:pr3.8.2}
    \frac{1}{2\sin\theta/2} = \sum_{l=0}^\infty (-1)^l P_l(\cos\theta)
  \end{gather}%
  となるので,2式\eqref{eq:pr3.8.1},\eqref{eq:pr3.8.2}の和を考えることで
  \begin{gather}%
    \frac{1}{2}\ab(\frac{1}{\sin\theta/2} + \frac{1}{\cos\theta/2}) = 2 \sum_{j=0}^\infty P_{2j}(\cos\theta)
  \end{gather}%
  が得られる.

  また,$(4j+1)/(2j+1) = 2 - 1/(2j+1)$であることに注意すれば
  \begin{gather}%
    \sigma(\theta) = -\frac{Q}{4\pi b^2} \ab[\frac{1}{2}\ab(\frac{1}{\sin\theta/2} + \frac{1}{\cos\theta/2}) - \sum_{j=0}^\infty \frac{1}{2j+1} P_{2j}(\cos\theta)]
  \end{gather}%
  も導かれる.


\clearpage
