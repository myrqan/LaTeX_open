\begin{enumerate}[label={1.\arabic*}]
  \item 
    \begin{enumerate}[(a)]
      \item 導体内部に電荷が存在するとどのような不都合が起こるかを考える.
      \item (前半)導体に厚みがあるとして,内部境界の任意の2点をとり,電場について
        線積分を実行するとよい.(後半)Gaussの法則を適用するだけ.
      \item 導体に接するような小さな箱を考えてGaussの法則を適用.
    \end{enumerate}
  \item $(x-x')^2 + (y-y')^2 + (z-z')^2 = \dl[2]{x} + \dl[2]{y} + \dl[2]{z}$として考えればよい.
  \item (a)--(d)通して特に言うことはない.積分した結果が,望んだ通りの答えになっていることを
    確認するとよい.
  \item Gaussの法則を適用するだけでよい.特に言うことはない.
  \item $\gr^2 (1/r) = -4\pi \delta^{(3)}(\bs{r})$であることに注意が必要(原点での$1/r$の取り扱い).
  \item 電場を計算して,それから電位差を求める.
  \item $a_1, a_2$の幾何平均は$a \equiv \sqrt{a_1 a_2}$で定義される.
  \item 
\end{enumerate}

