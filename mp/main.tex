%! TeX root = main.tex  
%% preamble for using:
%% - jlreq class
%% - b5paper, 8pt
%% -----------------------------------

\documentclass[%
  paper=b5paper,%
  fontsize=10pt,%
  jafontscale=0.875,%
  %landscape,%
  %twocolumn,%
  %fleqn,%
  %sidenote_length=4cm
]{jlreq}

\usepackage{geometry}
  \geometry{margin=1cm,bottom=1.5cm}

%%%
%%% package import
%%% make : Jul 9, 2025


% adjust in main_prb file(s)
%\usepackage{geometry}
%  \geometry{margin=2cm}

\usepackage{graphicx}

\usepackage[no-math,deluxe,expert,hiragino-pron]{luatexja-preset}


%% ---------------------------------
%% FOR MATHEMATIC EQUATIONS
%% ---------------------------------
\usepackage{amsmath,amssymb,amsthm}
\usepackage{stmaryrd}
\usepackage{siunitx}
\usepackage{physics2}
  \usephysicsmodule{ab}
  \usephysicsmodule{braket}
  \usephysicsmodule{diagmat}
  \usephysicsmodule{doubleprod}
  \usephysicsmodule{xmat}

\usepackage{mathtools}
\usepackage[ISO]{diffcoeff}
\usepackage[%
  math-style=TeX,%
  warnings-off={mathtools-colon,mathtools-overbracket}%
]{unicode-math}
\usepackage{tikz}
%% -----------------------------------

\usepackage{ascmac}
\usepackage{comment}
\usepackage{empheq}
\usepackage[shortlabels,inline]{enumitem}
\usepackage{fancybox,fancyhdr}
\usepackage{float}
\usepackage{lastpage}
\usepackage[listings]{tcolorbox}
	\tcbuselibrary{breakable}
	\tcbuselibrary{raster,skins}
%\usepackage{titlesec}
\usepackage{ulem}
\usepackage{url}
\usepackage{verbatim}
\usepackage{wrapfig}
\usepackage{wrapstuff}
\usepackage{varwidth}
\usepackage{color}
\usepackage[x11names]{xcolor}


\usepackage{fontspec}
\setmonofont{Inconsolatazi4} %% CHANGE DEFAULT TTFONT

\usepackage{listings,jvlisting}
  \lstset{
  basicstyle={\ttfamily\gtfamily},%
  identifierstyle={\ttfamily},%
  %commentstyle={\smallitshape},%
  keywordstyle={\ttfamily\bfseries},%
  %ndkeywordstyle={\small},%
  %stringstyle={\ttfamily},%
  frame={tblr},%
  breaklines=true,%
  columns=[l]{fullflexible},%
  numbers=left,%
  xrightmargin=0\zw,%
  xleftmargin=5ex,%
  xrightmargin=3ex,%
  numberstyle={\ttfamily\scriptsize},%
  numbersep=3ex,%
  lineskip=-1.ex,%
  backgroundcolor=\color{black!5!white},
  fontadjust=true
}

\usepackage[luatex,pdfencoding=auto]{hyperref}
  \hypersetup{%
    colorlinks = true,%
    linkcolor = green!50!black,%
    urlcolor = orange,%
    citecolor = Green,%
}


%%% CHANGE MATH FONT

\setmainfont{XITS}
\setmathfont{XITS Math}



\jlreqsetup{itemization_beforeafter_space=0pt}
\jlreqsetup{itemization_labelsep=1ex}
\jlreqsetup{caption_font=\small}
\jlreqsetup{caption_label_font=\small}

\ModifyHeading{section}{%
  font={\large\bfseries\gtfamily\sffamily},%
  after_space=0.7\baselineskip%
}
\ModifyHeading{subsection}{%
  font={\large\bfseries\gtfamily\sffamily},%
  after_space=0.7\baselineskip%
}
\ModifyHeading{subsubsection}{%
  font={\normalsize\bfseries\gtfamily\sffamily}%
}



\setlength{\columnseprule}{0.2pt}
\numberwithin{equation}{section}


%\renewcommand{\headrulewidth}{0.3pt} %% thickness of line (btwn header andy body).0pt = no line
\renewcommand{\figurename}{Fig.\,}
\renewcommand{\tablename}{Table\,}



\newcommand{\mctext}[1]{\mbox{\textcircled{\scriptsize{#1}}}}
\newcommand{\ctext}[1]{\textcircled{\scriptsize{#1}}}
\newcommand{\ds}{\displaystyle}
\newcommand{\comb}[2]{\left(\begin{matrix}#1\\#2\end{matrix}\right)}
\newcommand{\hs}{\hspace}
\newcommand{\qq}{\hspace{1em}}
\newcommand{\qqtext}[1]{\hspace{1em}\text{#1}\hspace{0.5em}}
\newcommand{\vs}{\vspace}
\newcommand{\emphvs}{\vspace{1em}\notag\\}
\newcommand{\ora}{\overrightarrow}
\newcommand{\oramr}[1]{\overrightarrow{\mathrm{#1}}}
\newcommand{\ol}{\overline}
\newcommand{\tri}{\triangle}
\newcommand{\mr}{\mathrm}
\newcommand{\mb}{\mathbb}
\newcommand{\mrvec}[1]{\overrightarrow{\mathrm{#1}}}
\newcommand{\itvec}{\overrightarrow}
%\newcommand{\bs}{\boldsymbol}
\newcommand{\bs}{\symbfit}%%symbfでdefine
\newcommand{\bsup}{\symbfup}

\newcommand{\ra}{\rightarrow}
\newcommand{\Ra}{\Rightarrow}
\newcommand{\lra}{\longrightarrow}
\newcommand{\Lra}{\Longrightarrow}
\newcommand{\la}{\leftarrow}
\newcommand{\La}{\Leftarrow}
\newcommand{\lla}{\longleftarrow}
\newcommand{\Lla}{\Longleftarrow}
\newcommand{\lr}{\leftrightarrow}
\newcommand{\llr}{\longleftrightarrow}
\newcommand{\Llr}{\Longleftrightarrow}
\newcommand{\mqty}[1]{\begin{matrix}#1\end{matrix}}
\newcommand{\avg}[1]{\left\langle{#1}\right\rangle}

\newcommand{\blue}[1]{\textcolor{blue}{#1}}
\newcommand{\red}[1]{\textcolor{red}{#1}}

\newcommand{\eval}[1]{\left.#1\right|}
\newcommand{\order}[1]{\mathcal{O}\ab(#1)}
\newcommand{\gr}{\nabla}
\newcommand{\di}{\nabla\cdot}
\newcommand{\ro}{\nabla\times}
\newcommand{\nb}{\nabla}



\renewcommand{\i}{\symup{i}}
\newcommand{\e}{\symup{e}}
\newcommand{\R}{\symbb{R}}
\newcommand{\N}{\symbb{N}}
\newcommand{\C}{\symbb{C}}
\newcommand{\al}{\alpha}
\newcommand{\be}{\beta}
\newcommand{\ga}{\gamma}
\newcommand{\de}{\delta}
\newcommand{\eps}{\varepsilon}
%\newcommand{\th}{\theta}


%% MATH OP.

\DeclareMathOperator{\GL}{GL}
\DeclareMathOperator{\SL}{SL}
\DeclareMathOperator{\diag}{diag}
\DeclareMathOperator{\tr}{tr}
\DeclareMathOperator{\sinc}{sinc}


%% COLOR DEFINITION

\definecolor{mycream}{HTML}{FEFAEA}
\definecolor{myvio}{HTML}{03155B}
\definecolor{blue-green}{rgb}{0.0, 0.87, 0.87}
\definecolor{bluegray}{rgb}{0.4, 0.6, 0.8}
\definecolor{mint}{rgb}{0.24, 0.71, 0.54}
\definecolor{dark-mint}{HTML}{3B876B}

%% DIFFCOEFF PKG SETTING

\difdef { l } { dn } { style = d^ }
\difdef { l } {} {outer-Rdelim = \,}
\difdef {f, s} {D}{op-symbol = \mathrm{D}}



\renewcommand{\contentsname}{Contents}
\AtBeginDocument{\addtocontents{toc}{\protect\thispagestyle{fancy}}}

%

\pagestyle{fancy}  
\fancyhead[L]{\leftmark}
\fancyhead[C]{}  
\fancyhead[R]{\textbf{\thepage}}  
\fancyfoot{}

%\renewcommand{\contentsname}{Contents}
%\renewcommand{\headrulewidth}{0.1pt}

\begin{document}
\begin{screen}
  \centering
  \Large
  物理数学に関するちょっとしたまとめ
\end{screen}
\tableofcontents
\hrulefill
\clearpage

\section{Fourier半区間展開について}
\subsection{一般のFourier展開}
区間$\ab[-\pi, \pi]$で(区分的に)連続な関数$f(x)$のFourier級数展開を考える.
のちに,これを一般化して,$\ab[-T, T\,]$で定義される関数$f(x)$についての級数展開も考える.
\begin{gather}  
  f(x) = \frac{a_0}{2} + \sum_{n=1}^\infty \ab(a_n\cos(nx) + b_n\sin(nx))
\end{gather}
ただし,係数$a_n, b_n$は
\begin{gather}
  \begin{dcases}
    a_n = \frac{1}{\pi} \int_{-\pi}^{\pi} \dl{x} f(x) \cos (nx) \qq\text{for}\qq n = 0, 1, 2, \ldots\\
    b_n = \frac{1}{\pi} \int_{-\pi}^{\pi} \dl{x} f(x) \sin (nx) \qq\text{for}\qq n = 1, 2, 3, \ldots\\
  \end{dcases}
\end{gather}
で与えられる.

特別な場合として,$f(x)$が偶関数のときは$n = 1, 2, 3, \ldots$に対して$b_n = 0$であり,
\begin{gather}
  a_n = \frac{2}{\pi} \int_0^\pi \dl{x} f(x) \cos(nx) \qq n = 0, 1, 2, \ldots
\end{gather}
また,$f(x)$が奇関数のときは,$n = 0, 1, 2, \ldots$に対して,$a_n = 0$であり,
\begin{gather}
  b_n = \frac{2}{\pi} \int_0^\pi \dl{x} f(x) \sin(nx) \qq n = 1, 2, 3, \ldots
\end{gather}
と書ける.

つぎに,周期が$2T$として,$\ab[-T, T\,]$で定義される$f(x)$の場合について考えるが,これは$n \to (\pi n)/T$として考えれば良い.係数を求めるための積分の区間は$-T\to T$とする.
\begin{gather}  
  f(x) = \frac{a_0}{2} + \sum_{n=1}^\infty \ab[a_n\cos\ab(\frac{\pi n}{T}x) + b_n \sin\ab(\frac{\pi n}{T}x)]
\end{gather}
\begin{gather}
  \begin{dcases}
    a_n = \frac{1}{T} \int_{-T}^T \dl{x} f(x) \cos\ab(\frac{\pi n}{T}x)\\
    b_n = \frac{1}{T} \int_{-T}^T \dl{x} f(x) \sin\ab(\frac{\pi n}{T}x)
  \end{dcases}
\end{gather}

\subsection{半区間でのFourier展開}
$[0, L]$で定義された関数$f(x)$をFourier級数展開することを考える.
\subsubsection{奇関数として拡張する場合}
\begin{wrapfigure}{r}{0.3\linewidth}
  \centering
  \includegraphics[width=\linewidth]{fig/odd_func.pdf}  
  \caption{奇関数の例}
\end{wrapfigure}
$f(x)$を奇関数として$[-L, L]$で定義したものを$g(x)$とする.
\begin{gather}  
  g(x) = 
  \begin{dcases}
    f(x) & (0 < x < L)\\
    -f(-x) & (-L<x<0)
  \end{dcases}
\end{gather}
のように区間$[-L, 0]$について拡張してやれば良い.
このようにして定義される$g(x)$を周期$2L$でFourier級数展開することで,
\begin{gather}
  g(x) = \sum_{n=1}^\infty b_n \sin\ab(\frac{\pi n}{L}x)
\end{gather}
ただし,
\begin{align}  
  b_n &= \frac{2}{L} \int_0^L \dl{x} g(x)\sin\ab(\frac{\pi n}{L}x)\\
      &= \frac{2}{L}\int_0^L \dl{x}f(x) \sin\ab(\frac{\pi n}{L} x)
\end{align}
と表すことができる.
\subsubsection{偶関数として拡張する場合}
\begin{wrapfigure}{r}{0.3\linewidth}
  \centering
  \includegraphics[width=\linewidth]{fig/even_func.pdf}  
  \caption{偶関数の例}
\end{wrapfigure}
$f(x)$を偶関数として$[-L, L]$で定義したものを$g(x)$とする.
\begin{gather}  
  g(x) = 
  \begin{dcases}
    f(x) & (0 < x < L)\\
    f(-x) & (-L<x<0)
  \end{dcases}
\end{gather}
$g(x)$をFourier展開すると,
\begin{gather}
  g(x) = \frac{a_0}{2} + \sum_{n=1}^\infty a_n \cos\ab(\frac{\pi n}{L}x)
\end{gather}
ただし,
\begin{align}  
  a_n &= \frac{2}{L} \int_0^L \dl{x} g(x) \cos\ab(\frac{\pi n}{L} x)\\
      &= \frac{2}{L} \int_0^L \dl{x} f(x) \cos\ab(\frac{\pi n}{L} x)
\end{align}
と表すことができる.



\end{document}
