\section{Bessel関数}
\subsection{Bessel関数,Neumann関数の導入}
Besselの微分方程式
\begin{gather}
  \label{eq:bessel_diffeq}
  \diff[2]{u(z)}{z} + \frac{1}{z} \diff{u(z)}{z} + \ab(1 - \frac{\nu^2}{z^2}) u(z) = 0
\end{gather}
を考える.いま,確定特異点$z = 0$周りの解をFrobeniusの方法で級数の形で解(のひとつ)を求めると,
\begin{gather}
  u(z) = J_\nu(z) \equiv \sum_{m=0}^{\infty} \frac{(-1)^m}{m! \Gamma(\nu+m+1)} \ab(\frac{z}{2})^{\nu+2m}
\end{gather}
を得る.これを,次数$\nu$のBesselの関数または第一種円筒関数(Bessel functions of the first kind)という.$\nu$を$-\nu$に置き換えても,同じBesselの微分方程式\eqref{eq:bessel_diffeq}を与えるから
\begin{gather}
  u(z) = J_{-\nu}(z) = \sum_{m=0}^{\infty} \frac{(-1)^m}{m! \Gamma(-\nu+m+1)} \ab(\frac{z}{2})^{-n+2m}
\end{gather}
も解(のひとつ)である.$\nu$が0以上の整数でない時は$J_\nu$と$J_{-\nu}$は線型独立な解であるから,一般解はこれらの線型結合で表すことができる.しかし,いま,$\nu$を0以上の整数とすると,
\begin{gather}
  J_{-n}(z) = (-1)^n J_n(z)
\end{gather}
なる関係が成り立ち,もはや$J_n$と$J_{-n}$は独立ではないので,$J_\nu$に線型独立な解を探す必要がある.このような解として
\begin{gather}
Y_\nu(z) = \frac{\cos(\nu \pi) J_\nu(z) - J_{-\nu}(z)}{\sin(\nu \pi)}
\end{gather}
を採用する.これを次数$\nu$の第二種円筒関数(Bessel functions of the second kind)またはNeumannの関数という.

従って,Besselの微分方程式\eqref{eq:bessel_diffeq}の一般解は線型独立な特解$J_\nu$と$Y_\nu$(とくに,$\nu \not \in \mb{Z}_{\geq 0}$のときは$J_{-n}$でもよい)の線形結合で表すことができる.

\subsection{Bessel関数の母関数表示}
Bessel関数の母関数が
\begin{gather}
  g(x, t) \equiv \exp\ab[\frac{z}{2}\ab(t - \frac{1}{t})] =  \sum_{n=-\infty}^{\infty} J_n(z) t^n
\end{gather}
で表されることを示す.
\begin{proof}
  $g(x, t)$を$t$についてLaurent展開して,
  \begin{gather}
    \exp\ab[\frac{z}{2}\ab(t-\frac{1}{t})] = \sum_{n=-\infty}^{\infty}A_nt^n
  \end{gather}
  のように表されるとする.このとき,留数定理より
  \begin{gather}
    A_n = \frac{1}{2\pi i} \oint_C \exp\ab[\frac{z}{2}\ab(t - \frac{1}{t})] \frac{\dl{t}}{t^{n+1}}
  \end{gather}
  である.ただし,積分曲線は原点を中心とする円であるとした.

  $t = 2u / z$と積分変数を変換し,さらにTaylor展開をすると,
  \begin{align}
    A_n &= \frac{1}{2\pi i }\sum_{m=0}^{\infty} \frac{(-1)^m}{m!} \ab(\frac{z}{2})^{2m+n} \sum_{s=0}^{\infty} \frac{1}{s!} \oint_C \frac{1}{u^{n+m-s+1}} \dl{u} \notag\\
        &= \sum_{m=0}^{\infty} \frac{(-1)^m}{m!} \ab(\frac{z}{2})^{2m+n} \sum_{s=0}^{\infty} \frac{1}{s!} \delta_{n+m, s} \notag\\
        &= \sum_{m=0}^{\infty} \frac{(-1)^m}{m! (n+m)!}\ab(\frac{z}{2})^{2m+n}
  \end{align}
  従って,$A_n = J_n$となるので,母関数表示を得ることができた.
\end{proof}

\subsection{円筒関数の基本関係}
第1種Bessel関数
\begin{gather}
  J_\nu(z) = \sum_{m=0}^{\infty} \frac{(-1)^m}{m! \Gamma(m+\nu+1)}\ab(\frac{z}{2})^{\nu+2m}
\end{gather}
に対して,簡単な計算により
\begin{gather}
  \label{eq:bessel}
  \left.
    \begin{aligned}
      2J_\nu'(z) &= J_{\nu-1}(z) - J_{\nu+1}(z) \\
      \frac{2\nu}{z}J_\nu(z) &= J_{\nu-1}(z) + J_{\nu+1}(z)
    \end{aligned}
  \right\}
\end{gather}
が得られる.
\eqref{eq:bessel}を一般化して,$J$を$\nu, z$の関数と見て$f(z, \nu)$と表すことにして,
\begin{gather}  
  \left.
  \begin{aligned}
    f(z, \nu-1) + f(z, \nu+1) &= \frac{2\nu}{z}f(z, \nu)\\
    f(x, \nu-1) - f(z, \nu+1) &= 2\diffp**{z}{f(z, \nu)}
  \end{aligned}
  \right\}
\end{gather}
を満たす$f$を変数$z$,次数$\nu$の円筒関数と言うこともある.

また,\eqref{eq:bessel}を組み合わせることで
\begin{gather}
  \left.
    \begin{aligned}
      \diff**{z}{\ab[z^\nu J_\nu(z)]} & = z^\nu J_{\nu-1}(z)\\
      \diff**{z}{\ab[z^{-\nu} J_{\nu}(z)]} & = -z^{-\nu} J_{\nu+1}(z)
    \end{aligned}
  \right\}
\end{gather}
が得られる.

