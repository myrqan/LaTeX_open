\section{Fourier半区間展開について}
\subsection{一般のFourier展開}
区間$\ab[-\pi, \pi]$で(区分的に)連続な関数$f(x)$のFourier級数展開を考える.
のちに,これを一般化して,$\ab[-T, T\,]$で定義される関数$f(x)$についての級数展開も考える.
\begin{gather}  
  f(x) = \frac{a_0}{2} + \sum_{n=1}^\infty \ab(a_n\cos(nx) + b_n\sin(nx))
\end{gather}
ただし,係数$a_n, b_n$は
\begin{gather}
  \begin{dcases}
    a_n = \frac{1}{\pi} \int_{-\pi}^{\pi} \dl{x} f(x) \cos (nx) \qq\text{for}\qq n = 0, 1, 2, \ldots\\
    b_n = \frac{1}{\pi} \int_{-\pi}^{\pi} \dl{x} f(x) \sin (nx) \qq\text{for}\qq n = 1, 2, 3, \ldots\\
  \end{dcases}
\end{gather}
で与えられる.

特別な場合として,$f(x)$が偶関数のときは$n = 1, 2, 3, \ldots$に対して$b_n = 0$であり,
\begin{gather}
  a_n = \frac{2}{\pi} \int_0^\pi \dl{x} f(x) \cos(nx) \qq n = 0, 1, 2, \ldots
\end{gather}
また,$f(x)$が奇関数のときは,$n = 0, 1, 2, \ldots$に対して,$a_n = 0$であり,
\begin{gather}
  b_n = \frac{2}{\pi} \int_0^\pi \dl{x} f(x) \sin(nx) \qq n = 1, 2, 3, \ldots
\end{gather}
と書ける.

つぎに,周期が$2T$として,$\ab[-T, T\,]$で定義される$f(x)$の場合について考えるが,これは$n \to (\pi n)/T$として考えれば良い.係数を求めるための積分の区間は$-T\to T$とする.
\begin{gather}  
  f(x) = \frac{a_0}{2} + \sum_{n=1}^\infty \ab[a_n\cos\ab(\frac{\pi n}{T}x) + b_n \sin\ab(\frac{\pi n}{T}x)]
\end{gather}
\begin{gather}
  \begin{dcases}
    a_n = \frac{1}{T} \int_{-T}^T \dl{x} f(x) \cos\ab(\frac{\pi n}{T}x)\\
    b_n = \frac{1}{T} \int_{-T}^T \dl{x} f(x) \sin\ab(\frac{\pi n}{T}x)
  \end{dcases}
\end{gather}

\subsection{半区間でのFourier展開}
$[0, L]$で定義された関数$f(x)$をFourier級数展開することを考える.
\begin{figure}[htbp]%  
  \centering%  
  \begin{minipage}{0.47\linewidth}
    \centering
    \includegraphics[width=\linewidth]{fig/odd_func.pdf}%  
    \caption{奇関数の例}
  \end{minipage}
  \begin{minipage}{0.47\linewidth}
    \centering
    \includegraphics[width=\linewidth]{fig/even_func.pdf}%  
    \caption{偶関数の例}
  \end{minipage}
\end{figure}%
\subsubsection{奇関数として拡張する場合}
$f(x)$を奇関数として$[-L, L]$で定義したものを$g(x)$とする.
\begin{gather}  
  g(x) = 
  \begin{dcases}
    f(x) & (0 < x < L)\\
    -f(-x) & (-L<x<0)
  \end{dcases}
\end{gather}
のように区間$[-L, 0]$について拡張してやれば良い.
このようにして定義される$g(x)$を周期$2L$でFourier級数展開することで,
\begin{gather}
  g(x) = \sum_{n=1}^\infty b_n \sin\ab(\frac{\pi n}{L}x)
\end{gather}
ただし,
\begin{align}  
  b_n &= \frac{2}{L} \int_0^L \dl{x} g(x)\sin\ab(\frac{\pi n}{L}x)\\
      &= \frac{2}{L}\int_0^L \dl{x}f(x) \sin\ab(\frac{\pi n}{L} x)
\end{align}
と表すことができる.
\subsubsection{偶関数として拡張する場合}
$f(x)$を偶関数として$[-L, L]$で定義したものを$g(x)$とする.
\begin{gather}  
  g(x) = 
  \begin{dcases}
    f(x) & (0 < x < L)\\
    f(-x) & (-L<x<0)
  \end{dcases}
\end{gather}
$g(x)$をFourier展開すると,
\begin{gather}
  g(x) = \frac{a_0}{2} + \sum_{n=1}^\infty a_n \cos\ab(\frac{\pi n}{L}x)
\end{gather}
ただし,
\begin{align}  
  a_n &= \frac{2}{L} \int_0^L \dl{x} g(x) \cos\ab(\frac{\pi n}{L} x)\\
      &= \frac{2}{L} \int_0^L \dl{x} f(x) \cos\ab(\frac{\pi n}{L} x)
\end{align}
と表すことができる.
