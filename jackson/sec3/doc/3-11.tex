円筒座標系でGreen関数の満たすべき式;
\begin{gather}
   \label{eq:Greenfunc_in_cylindar}
  \gr^2_x G(\bs{x}, \bs{x}') = -\frac{4\pi}{\rho} \delta(\rho - \rho') \delta(\phi - \phi') \delta (z - z')
\end{gather}
$\phi$と$z$に関する部分のデルタ関数を
\begin{gather}
  \delta(\phi - \phi') = \frac{1}{2\pi} \sum_{m=-\infty}^{\infty} \e^{\i m(\phi - \phi')}\\
\delta(z - z') = \frac{1}{2\pi} \int_\infty^\infty \dl{k} \e^{\i k(z-z')}  = \frac{1}{\pi} \int_0^\infty \dl{k}\cos\ab[k(z-z')]
\end{gather}
として,Green関数が
\begin{gather}  
  G(\bs{x}, \bs{x}') = \frac{1}{2\pi^2} \sum_{m=-\infty}^{\infty} \int_0^\infty \dl{k} \e^{\i m(\phi-\phi')} \cos \ab[k(z - z')] g_m(k; \rho, \rho')
\end{gather}
として表されるとする.

Green関数の式\eqref{eq:Greenfunc_in_cylindar}に代入して,丁寧に計算をすると,$g_m$の部分についての関係式が得られる;
\begin{gather}
  \label{eq:Greenfunc_cylindar_radius}
  \frac{1}{\rho} \diff**{\rho}{\ab(\rho\diff{g_m}{\rho})} - \ab(k^2 + \frac{m^2}{\rho^2})g_m = -\frac{4\pi}{\rho} \delta (\rho - \rho')
\end{gather}
$\rho \neq \rho'$のときは右辺がゼロになり,これはmodified Bessel functionである.一般解は$I_m(k\rho)$と$K_m(k\rho)$の線型結合で表される.

$\psi_1(k\rho)$と$\psi_2(k\rho)$がそれぞれ,$\rho < \rho'$,$\rho > \rho'$での境界条件を満たす,$I_m, K_m$の線型結合で表される(線型独立な)解であるとすると,Green関数の対称性($\rho$と$\rho'$の入れ替えに関して同じ関数を与えること)より,
\begin{gather}
  \label{eq:Greenfunc_cylindar_radius_psi}
  g_m(k; \rho, \rho') = \psi_1(k\rho_<)\psi_2(k\rho_>)
\end{gather}
で表すことができる.

$\rho = \rho'$での接続条件を考えると,
\begin{gather}  
  \eval{\diff{g_m}{\rho}}_{\rho = \rho' +0} - \eval{\diff{g_m}{\rho}}_{\rho = \rho' - 0} = - \frac{4\pi}{\rho'}
\end{gather}
となる.
これに,\eqref{eq:Greenfunc_cylindar_radius_psi}を代入すると,
\begin{gather}
  k W\ab[\psi_1(k\rho), \psi_2(k\rho)] = -\frac{4\pi}{\rho} \\
  \label{eq:wronskian_cylindar}
  \text{i.e.}\qq W\ab[\psi_1(x), \psi_2(x)] = -\frac{4\pi}{x}
\end{gather}
となる.ただし,$W\ab[\psi_1, \psi_2]$は$\psi_1$と$\psi_2$のWronskianであり,
\begin{gather}
  W\ab[\psi_1, \psi_2] = \psi_1 \psi_2' - \psi_1'\psi_2
\end{gather}
で与えられる.

\eqref{eq:Greenfunc_cylindar_radius}はStrum-Liouville型の微分方程式;
\begin{gather}
  \diff**{x}{\ab[p(x) \diff{y}{x}]} + g(x) y = 0
\end{gather}
である.
この方程式の線型独立な2つの解のWronskianは$1/p(x)$に比例する形でかけることが知られている.
このことを認めることにする.\footnote{いつかメモにしたい}

いま,境界面がない自由空間を考える.
$g_m$は$\rho = 0$で有限かつ$\rho \to \infty$でゼロになるので,(係数を$\psi_1$に取り込むことにすると)$\psi_1(k\rho) = AI_m(k\rho)$および,$\psi_2(k\rho) = K_m(k\rho)$と表される.係数$A$はWronskianの条件から定める.

Wronskianは$1/\rho$に比例しているが,これはどの$\rho$についても成立している.いま,$\rho \gg 1$の極限で考えることにすると,($\rho \ll 1$の極限で考えても良い)
modified Bessel functionの漸近形
\begin{gather}  
  \begin{dcases}
    I_m(x) \sim \frac{1}{\sqrt{2\pi x}}\e^x\\
    K_m(x) \sim \sqrt{\frac{\pi}{2x}}\e^{-x}
  \end{dcases}
\end{gather}
を代入して計算すれば,
\begin{gather}
  W \ab[I_m(x), K_m(x)] = -\frac{1}{x}
\end{gather}
となる,したがって,\eqref{eq:wronskian_cylindar}と係数を比較することにより,$A = 4\pi$を得る.

さて,自由空間におけるGreen関数は$G(\bs{x}, \bs{x}') = 1 / \ab|\bs{x} - \bs{x}'|$であったから,この結果より,
\begin{gather}
  \frac{1}{\ab|\bs{x} - \bs{x}'|} = \frac{2}{\pi} \sum_{m=-\infty}^{\infty} \int_0^\infty \dl{k} \e^{\i m(\phi - \phi')} \cos\ab[k(z - z')] I_m(k\rho_<) K_m(k\rho_>)
\end{gather}
これは,
実関数だけを用いて表すことができて,
\begin{multline}
  \frac{1}{\ab|\bs{x} - \bs{x}'|} = \frac{4}{\pi} \int_0^\infty \dl{k} \cos\ab[k(z - z')] \\
  \times \ab\{\frac{1}{2}I_0(k\rho_<)K_0(k\rho_>) + \sum_{m=1}^\infty \cos\ab[m(\phi - \phi')] I_m(k\rho_<) K_m(k\rho_>)\}
\end{multline}
と書くことができる.

$\bs{x}' = 0$としたとき,$m \geq 1$に対して,$I_m(0) = 0$であるから,$m = 0$の項だけが残り,($I_m(z) \sim (z/2)^m $ if $z \ll 1$である.)
\begin{gather}
  \label{eq:cosK0}
  \frac{1}{\sqrt{\rho^2 + z^2}} = \frac{2}{\pi} \int_0^\infty \dl{k} \cos(kz) K_0(k\rho)
\end{gather}
となる.
また,$\rho^2 \to R^2 = \rho^2 + (\rho')^2 - 2 \rho \rho' \cos(\phi - \phi')$と置き直すと, このとき,$|\bs{x} - \bs{x}'(z'=0)| = \sqrt{R^2 + z^2}$であるから,
\begin{gather}
  \frac{1}{\sqrt{R^2 + z^2}} = \frac{4}{\pi} \int_0^\infty \dl{k} \cos(kz) \ab\{ \frac12 I_0(k\rho_<)K_0(k\rho_>) + \sum_{m=1}^{\infty}\cos\ab[m(\phi - \phi')]I_m(k\rho_<) K_m(k\rho_>)\}
\end{gather}
これと\eqref{eq:cosK0}を比較して,
\begin{multline}
  \label{eq:k0sqrt}
  K_0\ab(k\sqrt{\rho^2 + (\rho')^2 - 2 \rho \rho'\cos(\phi - \phi')}) \\
  = I_0(k\rho_<)K_0(k\rho_>) + 2 \sum_{m=1}^{\infty}\cos\ab[m(\phi - \phi')]I_m(k\rho_<) K_m(k\rho_>)
\end{multline}
を得る.ここで,$k\to 0$の極限を考える.$z \ll 1$のとき,
\begin{align}
  I_m(z) &\sim \frac{1}{\Gamma(m+1)} \ab(\frac{z}{2})^m\\
  K_m(z) &\sim
  \begin{dcases}
    -\log\ab(\frac{z}2) - 0.5772\ldots \qq (m=0)\\
    \frac{\Gamma(m)}{2}\ab(\frac{2}{z})^m\qq (m\neq 0)
  \end{dcases}
\end{align}
で漸近形が与えられるので,これを\eqref{eq:k0sqrt}に代入して計算をすると,
\begin{gather}
  \log\ab(\frac{1}{\rho^2 + (\rho')^2 - 2 \rho \rho' \cos(\phi - \phi')}) = 2\log\ab(\frac{1}{\rho_>}) + 2 \sum_{m=1}^{\infty}\frac{1}{m}\ab(\frac{\rho_<}{\rho_>})^m\cos[m(\phi - \phi')]
\end{gather}
これは二次元極座標系におけるGreen関数の表式である.(問題2.17)

%%2025-01-24 書き終え 3.11

