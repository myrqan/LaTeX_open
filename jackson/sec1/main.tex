%! TeX root = main.tex
\input{/Users/crutont/Documents/github/LaTeX_open/prb/prb_20250103_a4.tex}
\pagestyle{fancy}  
\fancyhead[L]{}  
\fancyhead[C]{}  
\fancyhead[R]{\textbf{\thepage}}  
\fancyfoot{}

\renewcommand{\contentsname}{Contents}

\begin{document}  
\begin{flushright}  
  @crutont
\end{flushright}
\begin{screen}
  \centering
  J. D. Jackson, Classical Electrodynamics

  Section 1: Introduction to Electrostatics ノート
\end{screen}

\setcounter{section}{1}
\tableofcontents
\hrulefill

\subsection{Coulomb's Law}
Coulombの実験により,その大きさに比べて十分離れた位置に置かれた二つの帯電した電荷が受ける力について,以下のことが実験的に示された:
\begin{itemize}  
  \item それぞれの電荷の大きさによって異なる
  \item それらの距離の二乗に反比例
  \item その力は電荷を結ぶ直線上に向く
  \item 帯電体が異なる種類の電荷をもてば引力,同じ種類の電荷をもてば斥力が働く
\end{itemize}
ある一つの帯電体に周囲の他の小さな帯電体が及ぼす力はCoulombの二体間に働く力のベクトル的な和として表されることが実験的に示された.

\subsection{Electric Field}

\subsection{Gauss's Law}

\subsection{Differential Form of Gauss's Law}
\subsection{Another Equation of Electrostatics and the Scalar Potential}
\subsection{Surface Distributions of Charges and Dipoles and Discontinuities in the Electric Field and Potential}
\subsection{Poisson and Laplace Equations}
\subsection{Green's Theorem}
\subsection{Uniqueness of the Solution with Dirichlet or Neumann Boundary Conditions}
\subsection{Formal Solution of Electrostatic Boundary-Value Problem with Green Function}
\subsection{Electrostatic Potential Energy and Energy Density; Capacitance}
\subsection{Variational Approach to the Solution of the Laplace and Poisson Equations}
\subsection{Relaxation Method for Two-Dimensional Electrostatic Problems}



\end{document}

