%% prb :: version 2
%% japanese font :: IPA Mincho
%% english font :: ComputerModern
%% type :: LuaLaTeX

\documentclass[a5paper]{ltjsarticle}

\usepackage{geometry}
  \geometry{margin=1cm, top=1.5cm}
\usepackage{amsmath, amssymb, amsthm}
\usepackage{tikz}

\usepackage{ascmac} % boxnote, itembox, mask, screen, shadebox
\usepackage{comment}
\usepackage{empheq}
\usepackage[shortlabels, inline]{enumitem}
\usepackage{fancybox} % doublebox, ovalbox, fbox, sbox
\usepackage{fancyhdr}

\usepackage{here}
\usepackage{lastpage}
\usepackage{listings, jvlisting}

\usepackage{stmaryrd} % special symbols
\usepackage[listings]{tcolorbox}
\usepackage{titlesec}

\usepackage{ulem} %% uline, uuline, uwave, sout, xout
\usepackage{url} %% url
\usepackage{verbatim} %%\verb|hoge| or verbatim env. option[*] : space visible
\usepackage{wrapfig}
\usepackage{xcolor}
\usepackage{varwidth}

\usepackage{siunitx}
\usepackage{physics2}
\usephysicsmodule{ab}
\usephysicsmodule{braket}
\usephysicsmodule{diagmat}
\usephysicsmodule{doubleprod}
\usephysicsmodule{xmat}

\usepackage[ISO]{diffcoeff} %% detail setting in renew_newcmd_ver2.tex
\usepackage[math-style=TeX]{unicode-math}
\unimathsetup{math-style=TeX}
\setmathfont{latinmodern-math.otf}%default
\setmathfont{xcharter-math.otf}[range=\symbb]%blackboard bold
\usepackage{mathtools}
\usepackage[no-math,deluxe,expert]{luatexja-preset}

\usepackage[luatex,pdfencoding=auto]{hyperref}
\hypersetup{%
	colorlinks = true,%
	%linkcolor = dark-mint,%
	linkcolor = blue,%
	urlcolor = orange,%
}


%% font settings

%\usepackage{CormorantGaramond}
%\setmathfont{Garamond-Math.otf}

\setmainjfont{Harano Aji Mincho}[%
	BoldFont = {Harano Aji Mincho Bold},%
	Scale = 0.85%
]

\numberwithin{equation}{section}

%% TITLEFORMAT SETTING
\titleformat{\section}[block]{\filleft\large}{\uline{\LARGE\bfseries\thesection.\hspace{.5em}}}{0em}{\uline}
\titleformat{\subsection}{\normalsize}{\uline{\large\bfseries\thesubsection.\hspace{.5em}}}{0em}{\uline}
\titleformat{\subsubsection}{\normalsize}{\bfseries\thesubsubsection.\hspace{.5em}}{0em}{}
\renewcommand{\thesection}{\arabic{section}}
\renewcommand{\thesubsection}{\arabic{section}.\arabic{subsection}}
\renewcommand{\thesubsubsection}{\arabic{section}.\arabic{subsection}.\arabic{subsubsection}}
%\renewcommand{\thesubsubsection}{\fontspec[Scale=0.8]{Futura Medium}\arabic{section}.\arabic{subsection}.\arabic{subsubsection}}
\titlespacing{\subsection}{0pt}{5pt}{5pt}
\titlespacing{\subsubsection}{0pt}{3pt}{5pt}


%% THEOREMSTYLE
\theoremstyle{definition}
\newtheorem{dfn}{Definition}[section]
\newtheorem{prop}[dfn]{Proposition}
\newtheorem{lem}[dfn]{Lemma}
\newtheorem{thm}[dfn]{Theorem}
\newtheorem{cor}[dfn]{Corollary}
\newtheorem{rem}[dfn]{Remark}
\newtheorem{fact}[dfn]{Fact}
% TCBUSELIBRARY
\tcbuselibrary{most}
\tcbset{colback=white,colframe=blue!75!black}
\newtcbtheorem[number within = section]{example}{Example}{
    enhanced,
    %colback = white
    colframe = blue!50!white,
    %fonttitle=\bfseries,
    breakable = true
  }{thm}
  %enhanced, breakable=true, colframe=blue!75!black, colback=blue!5!white, title={Exemple}, arc = 0mm}
  %{colback=green!5,colframe=green!35black,fonttitle=\bfseries}{th}

\newtcbtheorem[number within = section]{exercise}{Exercise}{
  enhanced, 
  breakable=true,
  colframe=yellow!25!green
  %colback=white
  }{thm}

\newtcbtheorem[use counter from=example]{theorem}{Theorem}{
  enhanced, 
  breakable=true, 
  colframe=green!30!blue
  %colback=white
  }{thm}

\newtcbtheorem[use counter from=example]{proposition}{Proposition}{
  enhanced, 
  breakable=true, 
  colframe=orange!75!black,
  %colback=white
  }{thm}

\newtcbtheorem[use counter from=example]{definition}{Definition}{
  enhanced, 
  breakable=true, 
  colframe=magenta!30!green,
  %colback=white
}{thm}
\newtcbtheorem[use counter from=example]{remark}{Remark}{
  enhanced,
  breakable=true, 
  colframe=red!80!white, 
  %colback=white
}{thm}


\input{/Users/crutont/Documents/github/LaTeX_open/prb/renew_newcmd_ver2.tex}
