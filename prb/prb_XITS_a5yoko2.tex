%% prb :: version 2
%% japanese font :: Zen Old Mincho Regular / Medium
%% english font :: STIX
%% type :: LuaLaTeX

%\documentclass[a5paper,8pt,landscape,twocolumn,fleqn]{ltjsarticle}
\documentclass[paper=a5paper,fontsize=8pt,landscape,twocolumn,fleqn]{jlreq}

\usepackage{geometry}
  \geometry{margin=0.5cm,top=1cm}
\usepackage{amsmath, amssymb, amsthm}
\usepackage{tikz}

\usepackage{ascmac} % boxnote, itembox, mask, screen, shadebox
%\usepackage{caption}
\usepackage{comment}
\usepackage{empheq}
\usepackage[shortlabels, inline]{enumitem}
\usepackage{fancybox} % doublebox, ovalbox, fbox, sbox
\usepackage{fancyhdr}

\usepackage{here}
\usepackage{lastpage}
\usepackage{listings, jvlisting}

\usepackage{stmaryrd} % special symbols
\usepackage[listings]{tcolorbox}
\usepackage{titlesec}

\usepackage{ulem} %% uline, uuline, uwave, sout, xout
\usepackage{url} %% url
\usepackage{verbatim} %%\verb|hoge| or verbatim env. option[*] : space visible
\usepackage{wrapfig}
\usepackage[svgnames]{xcolor}
\usepackage{varwidth}
\usepackage{wrapstuff}

\usepackage{siunitx}
\usepackage{physics2}
\usephysicsmodule{ab}
\usephysicsmodule{braket}
\usephysicsmodule{diagmat}
\usephysicsmodule{doubleprod}
\usephysicsmodule{xmat}

\usepackage[ISO]{diffcoeff} %% detail setting in renew_newcmd_ver2.tex
\usepackage[math-style=TeX]{unicode-math}
\unimathsetup{math-style=TeX}

\setmainfont{XITS}
\setmathfont{XITS Math}

\usepackage{mathtools}
\usepackage[no-math,deluxe,expert]{luatexja-preset}

\usepackage[luatex,pdfencoding=auto]{hyperref}
\hypersetup{%
	colorlinks = true,%
	linkcolor = dark-mint,%
	%linkcolor = blue,%
	urlcolor = orange,%
	citecolor = Green,%
}


%% font settings

%\usepackage{CormorantGaramond}
%\setmathfont{Garamond-Math.otf}

%\setmainjfont{Harano Aji Mincho}[%
%	BoldFont = {Harano Aji Gothic Bold},
%	%BoldFont = {Harano Aji Mincho Bold},%
%	Scale = 0.87%
%]
\setmainjfont{Toppan Bunkyu Mincho Regular}[
	Scale=0.90
]
%\setmainjfont{Zen Old Mincho Regular}[
%BoldFont = {Zen Old Mincho Medium},
%Scale=0.9
%]

%\setmainjfont{Toppan Bunkyu Mincho Regular}

%\setmainfont[%
%ItalicFont=NewCM10-Italic.otf,%
%BoldFont=NewCM10-Bold.otf,%
%BoldItalicFont=NewCM10-BoldItalic.otf,%
%SmallCapsFeatures={Numbers=OldStyle}]{NewCM10-Regular.otf}
%\setmathfont{Theano Modern Regular}[range={"0030-"0039}]

%\setmainfont[%
%ItalicFont=NewCM10-Italic.otf,%
%BoldFont=NewCM10-Bold.otf,%
%BoldItalicFont=NewCM10-BoldItalic.otf,%
%SmallCapsFeatures={Numbers=OldStyle}]{NewCM10-Regular.otf}

%\setsansfont[%
%ItalicFont=NewCMSans10-Oblique.otf,%
%BoldFont=NewCMSans10-Bold.otf,%
%BoldItalicFont=NewCMSans10-BoldOblique.otf,%
%SmallCapsFeatures={Numbers=OldStyle}]{NewCMSans10-Regular.otf}

%\setmonofont[ItalicFont=NewCMMono10-Italic.otf,%
%BoldFont=NewCMMono10-Bold.otf,%
%BoldItalicFont=NewCMMono10-BoldOblique.otf,%
%SmallCapsFeatures={Numbers=OldStyle}]{NewCMMono10-Regular.otf}

%\setmathfont{NewCMMath-Regular.otf}
%\setmathfont{STIXTwoMath.otf}
%\setmainfont{STIXTwoText-Regular}[
%    Extension=.otf,
%    BoldFont=STIXTwoText-Bold,
%    ItalicFont=STIXTwoText-Italic,
%    BoldItalicFont=STIXTwoText-BoldItalic
%]

\numberwithin{equation}{section}

%% TITLEFORMAT SETTING
%\titleformat{\section}[block]{\filleft\large}{\uline{\LARGE\thesection.\hspace{.5em}}}{0em}{\uline}
%\titleformat{\subsection}{\normalsize}{\uline{\large\bfseries\thesubsection.\hspace{.5em}}}{0em}{\uline}
%\titleformat{\subsubsection}{\normalsize}{\uline{\bfseries\thesubsubsection.}\hspace{.5em}}{0em}{}

%\titleformat{\section}[block]{\filleft\large}{\uline{\LARGE\thesection.\hspace{.5em}}}{0em}{\uline}
\titleformat{\section}[block]{\large}{\uline{\LARGE\thesection.\hspace{.5em}}}{0em}{\uline}
\titleformat{\subsection}{\normalsize}{\uline{\Large\thesubsection.\hspace{.5em}}}{0em}{\uline}
\titleformat{\subsubsection}{\normalsize}{\uline{\large\thesubsubsection.}\hspace{.5em}}{0em}{}
\titleformat{\paragraph}{\normalfont\normalsize}{\theparagraph.}{1em}{}



\renewcommand{\thesection}{\arabic{section}}
\renewcommand{\thesubsection}{\arabic{section}.\arabic{subsection}}
\renewcommand{\thesubsubsection}{\arabic{section}.\arabic{subsection}.\arabic{subsubsection}}
%\renewcommand{\thesubsubsection}{\fontspec[Scale=0.8]{Futura Medium}\arabic{section}.\arabic{subsection}.\arabic{subsubsection}}
\titlespacing{\subsection}{0pt}{5pt}{5pt}
\titlespacing{\subsubsection}{0pt}{3pt}{5pt}
%\titlespacing*{\paragraph}{0pt}{3.25ex plus 1ex minus .2ex}{1.5ex plus .2ex}
\titlespacing*{\paragraph}{0pt}{3pt}{5pt}

%% THEOREMSTYLE
\theoremstyle{definition}
\newtheorem{dfn}{Definition}[section]
\newtheorem{prop}[dfn]{Proposition}
\newtheorem{lem}[dfn]{Lemma}
\newtheorem{thm}[dfn]{Theorem}
\newtheorem{cor}[dfn]{Corollary}
\newtheorem{rem}[dfn]{Remark}
\newtheorem{fact}[dfn]{Fact}
% TCBUSELIBRARY
\tcbuselibrary{most}
\tcbset{colback=white,colframe=blue!75!black}
\newtcbtheorem[number within = section]{example}{Example}{
    enhanced,
    %colback = white
    colframe = blue!50!white,
    %fonttitle=\bfseries,
    breakable = true
  }{thm}
  %enhanced, breakable=true, colframe=blue!75!black, colback=blue!5!white, title={Exemple}, arc = 0mm}
  %{colback=green!5,colframe=green!35black,fonttitle=\bfseries}{th}

\newtcbtheorem[number within = section]{exercise}{Exercise}{
  enhanced, 
  breakable=true,
  colframe=yellow!25!green
  %colback=white
  }{thm}

\newtcbtheorem[use counter from=example]{theorem}{Theorem}{
  enhanced, 
  breakable=true, 
  colframe=green!30!blue
  %colback=white
  }{thm}

\newtcbtheorem[use counter from=example]{proposition}{Proposition}{
  enhanced, 
  breakable=true, 
  colframe=orange!75!black,
  %colback=white
  }{thm}

\newtcbtheorem[use counter from=example]{definition}{Definition}{
  enhanced, 
  breakable=true, 
  colframe=magenta!30!green,
  %colback=white
}{thm}
\newtcbtheorem[use counter from=example]{remark}{Remark}{
  enhanced,
  breakable=true, 
  colframe=red!80!white, 
  %colback=white
}{thm}


\jlreqsetup{caption_font=\small}
\jlreqsetup{caption_label_font=\small}
\jlreqsetup{caption_align={center,*left}}

\input{/Users/crutont/Documents/github/LaTeX_open/prb/renew_newcmd_ver2.tex}

\setlength{\columnseprule}{0.3pt}

\renewcommand{\labelitemi}{$\smwhtlozenge\,$}
\renewcommand{\labelitemii}{$\smblklozenge\,$}
\renewcommand{\labelitemiii}{$\diamond\,$}
\renewcommand{\labelitemiv}{$\cdot\,$}
